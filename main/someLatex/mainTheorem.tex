
\begin{enumerate}[(i)]
    \item
        设Bernstein-von Mises定理(见\cite{van2000asymptotic})的条件满足: $\Theta$是$\mathbb{R}^p$的开子集,试验($P_{\theta}:\theta\in\Theta$)在$\theta_0\in \Theta$处均方可微,具有非奇异信息阵$I_{\theta_0}$,假设对任意的$\epsilon>0$,存在一个相合检验序列:
        \begin{equation}
            P_{\theta_0}^n\phi_n\to 0,\quad \sup_{\|\theta-\theta_0\|\geq \epsilon} P_\theta^n(1-\phi_n)\to 0.
        \end{equation}
        设$\pi_n(h;X)$是一个满足Bernstain-von Mises定理结论的权函数:
        \begin{equation}
            \|\pi_n(h;X)-dN(\Delta_{n,\theta_0},I_{\theta_0}^{-1})(h)\|\overset{P_{\theta_0}^n}{\to}0
        \end{equation}
    \item
        %$\pi_n(h;X)$的支撑在$\|h\|\leq K\sqrt{n}$上,$K$是一个固定常数。
        
        

        For every $\epsilon>0$, there's a Lebesgue integrable function $T(h)$, a $K>0$ and a $A>0$ such that 

\begin{equation}
    \lim_{n\to \infty}P_{\theta_0}^n(\sup_{\|h\|\geq K\sqrt{n}}(\pi_n(h;X)-T(h))\leq 0)\geq 1-\epsilon
\end{equation}

        \begin{equation}
            \lim_{n\to \infty} P_{\theta_0}^n(\sup_{\|h\|\leq K\sqrt{n}} \pi_n(h;X)\leq A)\geq 1-\epsilon
        \end{equation}


    \item
        存在$\theta_0$的某个邻域$V$和某个函数$\dot{\ell}$满足$P_{\theta_0}\dot{\ell}^2<\infty$,对$\forall \theta_1,\theta_2\in V$,有
        \begin{equation}
            |\log p_{\theta_1}(x)-\log p_{\theta_2}(x)|\leq \dot{\ell}(x)\|\theta_1-\theta_2\|.
        \end{equation}
\end{enumerate}


\begin{theorem}\label{theoremMain}
    Under the Assumptions $(i)$, $(ii)$ and $(iii)$. For bounded real numbers $\eta_n$, we have
    \begin{equation}
        \Big|\int_{\mathbb{R}^{p}}\frac{p_h(X)}{p_0(X)}\pi_n(h;X)\,dh-
        2^{-\frac{p}{2}}e^{\frac{1}{2}\Delta_{n,\theta_0}^TI_{\theta_0}\Delta_{n,\theta_0}}
        \Big|\xrightarrow{P_{\eta_n}^n}0
    \end{equation}
\end{theorem}


\begin{proof}
    By contiguity,we only need to proof the convergence in $P_0^n$.

The proof consists of two steps. In the first part of the proof, let $C$ be the ball of fixed radius $M$ around zero. We proof

\begin{equation}\label{eq:14}
    \left|\int_C \frac{p^n_h(X)}{p^n_0(X)}\pi_n (h;X) \, dh-\int_C e^{h^TI_{\theta_0}\Delta_{n,\theta_0}-\frac{1}{2}h^TI_{\theta_0}h}dN(\Delta_{n,\theta_0},I_{\theta_0}^{-1})(h)\, dh\right|
 \xrightarrow{P^n_0}0
\end{equation}
By Lemma \ref{lemmaUniform}, for every fixed $M$,
\begin{equation}
    \sup_{\|h\|\leq M}|\log \frac{p_h^n(X)}{p_0^n(X)}-h^TI_{\theta_0}\Delta_{n,\theta_0}+\frac{1}{2}h^TI_{\theta_0}h|\xrightarrow{P_0^n}0 
\end{equation}
Hence we have
\begin{equation}\label{eq:8}
    \int_C \frac{p_h^n(X)}{p_0^n(X)}\pi_n (h;X) \, dh=e^{o_{p^n_0}(1)}\int_C e^{h^TI_{\theta_0}\Delta_{n,\theta_0}-\frac{1}{2}h^TI_{\theta_0}h}\pi_n (h;X) \, dh
\end{equation}
So we only need to consider $\int_C e^{h^TI_{\theta_0}\Delta_{n,\theta_0}-\frac{1}{2}h^TI_{\theta_0}h}\pi_n (h;X) \, dh$. Notice that $C$ is a bounded region,hence $\Delta_{n,\theta_0}$ converges to a normal distribution. Therefore, $\sup_{h\in C}e^{h^TI_{\theta_0}\Delta_{n,\theta_0}-\frac{1}{2}h^TI_{\theta_0}h}$ is bounded in probability. So for every $\delta>0$, there exists $M$ such that, with probability $1-\delta$,
\begin{equation}
\begin{aligned}
    \int_C e^{h^TI_{\theta_0}\Delta_{n,\theta_0}-\frac{1}{2}h^TI_{\theta_0}h}|\pi_n (h;X)-dN(\Delta_{n,\theta_0},I_{\theta_0}^{-1})(h)|\, dh
\\
\leq M\int_C |\pi_n(h;X)-dN(\Delta_{n,\theta_0},I_{\theta_0}^{-1})(h)|\, dh\xrightarrow{P^n_0}0
\end{aligned}
\end{equation}
Combining with \eqref{eq:8}, we can conclude that \eqref{eq:14} holds. \\

This is true for every ball $C$ of fixed radius $M$ and hence also for some $M_n\to \infty$.

In the second part, we proof
\begin{equation}\label{eq:4}
    \frac{\int_{c_n^c}p_h^n(X)\pi_n(h;X)\, dh}{\int p_h^n(X)\pi_n(h;X)\, dh}\xrightarrow{P_0^n}0,
\end{equation}
where $C_n$ is a ball with radius $M_n$, for any $M_n\to \infty$.
    
    Let $\phi_n$ be the test satisfies condition $(i)$, we have

\begin{equation}
    \frac{\int_{C_n^C}p_h^n(x)\pi(h|x)\, dh}{\int p_h^n(x)\pi(h|x)\, dh}= \frac{\int_{C_n^C}p_h^n(x)\pi(h|x)\, dh}{\int p_h^n(x)\pi(h|x)\, dh}\phi_n+ \frac{\int_{C_n^C}p_h^n(x)\pi(h|x)\, dh}{\int p_h^n(x)\pi(h|x)\, dh}(1-\phi_n)
\end{equation}
Since $\eqref{eq:4}\leq 1$, 
\begin{equation}
    \frac{\int_{C_n^C}p_h^n(X)\pi_n(h;X)\, dh}{\int p_h^n(X)\pi_n(h;X)\, dh}\phi_n\leq \phi_n\xrightarrow{P_0^n}0
\end{equation}
It's enough to proof
\begin{equation}\label{eq:10}
    \frac{\int_{C_n^C}p_h^n(X)\pi_n(h;X)\, dh}{\int p_h^n(X)\pi_n(h;X)\, dh}(1-\phi_n)\xrightarrow{P_0^n}0
\end{equation}
Fix a ball $U$ around zero. Then
\begin{equation}\label{eq:11}
\eqref{eq:10}\leq \frac{\int_{C_n^C}p_h^n(X)\pi_n(h;X)\, dh}{\int_U p_h^n(X)\pi_n(h;X)\, dh}(1-\phi_n)
\end{equation}

    By the Assumption (ii) and the fact that $\Delta_{n,\theta_0}$ is uniformly tight, we can assume $\sup_h (\pi_n(h;X)-T(h))\leq 0$ and $|\Delta_{n,\theta_0}|\leq M$ for some $M$ without loss of generality since the probability they don't hold will be eventually smaller than any prespecified constant. 

    %由条件$(ii)$,给定$\delta$,当$n$充分大时,以概率$1-\delta$有$\pi_n(h;X)<A$($\forall h\in U$)。 给定$\delta_1$, 则$M_{\delta_1}$ 使得当$n$充分大时,以概率$1-\delta_1$有$|\Delta_{n,\theta_0}|<M_{\delta_1}$。
    There exists an $m>0$ such that
\begin{equation}
    \inf_{h\in U} dN(\Delta_{n,\theta_0},I_{\theta_0}^{-1})(h)\geq m.
\end{equation}
Let $D_n(X)$ be the set $\{h: |\pi_{n}(h;X)-n(\Delta_{n,\theta_{0}},I_{\theta_{0}}^{-1})|\geq\frac{m}{2}\}$. We have
%再注意到$\pi_n(h;X)$的支撑在$\{h:\|h\|\leq K\}$上。所以我们以概率$1-\delta-\delta_1$有
\begin{equation}\label{eq:13}
    \begin{aligned}
        \eqref{eq:11}\leq&\frac{\int_{C_n^C}p_h^n(X)\pi_n(h;X)\, dh}{\int_{U/D_n(X)} p_h^n(X)\pi_n(h;X)\, dh}(1-\phi_n)\\
        \leq&\frac{\int_{C_n^C}p_h^n(X)\pi_n(h;X)\, dh}{\frac{m}{2}\int_{U/D_n(X)} p_h^n(X)\, dh}(1-\phi_n)\\
    \end{aligned}
\end{equation}
Next we proof
\begin{equation}\label{eq:12}
    \frac{\int_{U/D_n(X)} p_h^n(X)\, dh}{\int_U p_h^n(X)\, dh}\xrightarrow{P_0^n}1.
\end{equation}
By Assumption $(i)$, Bernstein-von Mises Theorem holds. That is
\begin{equation}\label{eq:9}
 \int |\pi_{n}(h;X)-n(\Delta_{n,\theta_{0}},I_{\theta_{0}}^{-1})|\, dh\xrightarrow{P_0^n}0   
\end{equation}
\eqref{eq:9} implies $\int 1_{D_n(x)}\, dh\xrightarrow{P_0^n}0$. Similar to the proof in step 1, we have

\begin{equation}
\begin{aligned}
    \eqref{eq:12}&=\frac{\int_{U/D_n(X)} \frac{p_h^n(X)}{p_0^n(X)}\, dh}{\int_U \frac{p_h^n(X)}{p_0^n(X)}\, dh}\\
                 &=\frac{e^{o_{P^n_0}(1)}\int_{U/D_n(X)} e^{h^T I_{\theta_0}\Delta_{n,\theta_0}-\frac{1}{2}h^TI_{\theta_0}h} \, dh}{e^{o^n_{P_0}(1)}\int_U e^{h^T I_{\theta_0}\Delta_{n,\theta_0}-\frac{1}{2}h^TI_{\theta_0}h} \, dh}\\
                 &\xrightarrow{P_0^n} 1
\end{aligned}
\end{equation}
since $h^T I_{\theta_0}\Delta_{n,\theta_0}-\frac{1}{2}h^TI_{\theta_0}h$ is bounded.


Now, to obtain $\eqref{eq:13}\xrightarrow{P_0^n}0$, we only need to prove
\begin{equation}\label{eq:5}
    \begin{aligned}
        \frac{\int_{C_n^C}p_h^n(X)(A\textbf{1}_{M_n\leq \|h\|\leq K\sqrt{n}}+T(h)\textbf{1}_{\|h\|> K\sqrt{n}})\, dh}{\int_U p_h^n(X)\, dh}(1-\phi_n)\xrightarrow{P_0^n} 0
    \end{aligned}
\end{equation}
By Lemma~\ref{lemmaContiguity}, we only need to prove $\eqref{eq:5}\xrightarrow{P_U^n}0$. To prove that, we only need to prove $\eqref{eq:5}\xrightarrow{L^1_{P_U^n}}0$, that is 
\begin{equation}\label{eq:6}
    \int \frac{\int_{C_n^C}p_h^n(x)(A\textbf{1}_{M_n\leq \|h\|\leq K\sqrt{n}}+T(h)\textbf{1}_{\|h\|> K\sqrt{n}})\, dh}{\int_U p_h^n(x)\, dh}(1-\phi_n)\big(\int_U p_h^n(x)dh\big) \, d\mu  \to 0
\end{equation}
We note that
\begin{equation}\label{eq:7}
    \eqref{eq:6}=\int_{C_n^C} \Big(\int (1-\phi_n)p_h^n(x)d\mu\Big) (A\textbf{1}_{M_n\leq \|h\|\leq K\sqrt{n}}+T(h)\textbf{1}_{\|h\|> K\sqrt{n}})\, dh 
\end{equation}
By Lemma \ref{lemmaTest}, there automatically exist tests $\phi_n$  such that for sufficiently large $n$ and $\|h\|\geq M_n$,
\begin{equation}
\int (1-\phi_n)p^n_h(x)d\mu\leq e^{-c(\|h\|^2\wedge n)}
\end{equation}
If $K< 1$, then $\|h\|^2\wedge n\geq \|h\|^2\wedge K^2n$. If $K\geq 1$, then
\begin{equation}
    \|h\|^2\wedge n=\frac{1}{K^2}(K^2\|h\|^2\wedge K^2n)\geq \frac{1}{K^2}(\|h\|^2\wedge K^2n)
\end{equation}
Let $c^*=c\min(1,1/K^2)$, then
\begin{equation}
\int (1-\phi_n)p^n_h(x)d\mu\leq e^{-c^*(\|h\|^2\wedge K^2n)}
\end{equation}
Splitting the integral into the domains $M_n\leq \|h\|\leq K\sqrt{n}$ and $\|h\|\geq K\sqrt{n}$, we see that
\begin{equation}
    \eqref{eq:7}\leq \int_{\|h\|\geq M_n}e^{-c^*\|h\|^2}\, dh + e^{-c^*K^2n}\int_{\|h\|>K\sqrt{n}} T(h)\, dh  \to 0
\end{equation}
Finally we have
\begin{equation}
    \begin{aligned}
        &\left|\int \frac{p_h(X)}{p_0(X)}\pi_n (h;X) \, dh-2^{-\frac{p}{2}}e^{\frac{1}{2}\Delta_{n,\theta_0}^TI_{\theta_0}\Delta_{n,\theta_0}}
 \right|\\
        &=\left|\int \frac{p_h(X)}{p_0(X)}\pi_n (h;X) \, dh-\int_{C_n} \frac{p_h(X)}{p_0(X)}\pi_n (h;X) \, dh\right|\\
        &+\left|\int_{C_n} \frac{p_h(X)}{p_0(X)}\pi_n (h;X) \, dh -\int_{C_n} e^{h^TI_{\theta_0}\Delta_{n,\theta_0}-\frac{1}{2}h^TI_{\theta_0}h}dN(\Delta_{n,\theta_0},I_{\theta_0}^{-1})(h)\, dh\right|\\
        &+\left| \int_{C_n} e^{h^TI_{\theta_0}\Delta_{n,\theta_0}-\frac{1}{2}h^TI_{\theta_0}h}n(\Delta_{n,\theta_0},I_{\theta_0}^{-1})\, dh-2^{-\frac{p}{2}}e^{\frac{1}{2}\Delta_{n,\theta_0}^TI_{\theta_0}\Delta_{n,\theta_0}}
 \right|\\
        &=J_1+J_2+J_3
\end{aligned}
\end{equation}
由证明的第一步,我们有$J_2\xrightarrow{P^n_0}0$, 所以我们知道$\int_{C_n} \frac{p_h(X)}{p_0(X)}\pi_n (h;X) \, dh $ 依概率有界。所以
\begin{equation}
\begin{aligned}
    J_1&=\int_{C_n} \frac{p_h(X)}{p_0(X)}\pi_n (h;X) \, dh\left|\frac{\int \frac{p_h(X)}{p_0(X)}\pi_n (h;X) \, dh}{\int_{C_n} \frac{p_h(X)}{p_0(X)}\pi_n (h;X) \, dh}-1\right|\\
       &=O_{P_0^n}(1)o_{P_0^n}(1)
\end{aligned}
\end{equation}
And $J_3$ convenges to $0$ for trivial reasion.
\end{proof}
