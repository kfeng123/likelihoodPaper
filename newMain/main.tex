\documentclass[11pt]{article}







\usepackage{lineno,hyperref}

\usepackage[margin=1 in]{geometry}
\renewcommand{\baselinestretch}{1.25}

\usepackage{galois} % composition function \comp
\usepackage{bm}
\usepackage{amsmath}
\usepackage{amssymb}
\usepackage{mathrsfs}
\usepackage{amsthm}
\usepackage{natbib}
\usepackage{graphicx}
\usepackage{color}
\usepackage{booktabs}
\usepackage[page,title]{appendix}
%\renewcommand\appendixname{haha}
\usepackage{enumerate}
\usepackage{changepage}
\usepackage{datetime}
\newdate{date}{9}{1}{2017}

%%%%%%%%%%%%%%  Notations %%%%%%%%%%
\DeclareMathOperator{\mytr}{tr}
\DeclareMathOperator{\mydiag}{diag}
\DeclareMathOperator{\myrank}{Rank}
\DeclareMathOperator{\myP}{P}
\DeclareMathOperator{\myE}{E}
\DeclareMathOperator{\myVar}{Var}
\DeclareMathOperator*{\argmax}{arg\,max}
\DeclareMathOperator*{\argmin}{arg\,min}


\newcommand{\Ba}{\mathbf{a}}    \newcommand{\Bb}{\mathbf{b}}    \newcommand{\Bc}{\mathbf{c}}    \newcommand{\Bd}{\mathbf{d}}    \newcommand{\Be}{\mathbf{e}}    \newcommand{\Bf}{\mathbf{f}}    \newcommand{\Bg}{\mathbf{g}}    \newcommand{\Bh}{\mathbf{h}}    \newcommand{\Bi}{\mathbf{i}}    \newcommand{\Bj}{\mathbf{j}}    \newcommand{\Bk}{\mathbf{k}}    \newcommand{\Bl}{\mathbf{l}}
\newcommand{\Bm}{\mathbf{m}}    \newcommand{\Bn}{\mathbf{n}}    \newcommand{\Bo}{\mathbf{o}}    \newcommand{\Bp}{\mathbf{p}}    \newcommand{\Bq}{\mathbf{q}}    \newcommand{\Br}{\mathbf{r}}    \newcommand{\Bs}{\mathbf{s}}    \newcommand{\Bt}{\mathbf{t}}    \newcommand{\Bu}{\mathbf{u}}    \newcommand{\Bv}{\mathbf{v}}    \newcommand{\Bw}{\mathbf{w}}    \newcommand{\Bx}{\mathbf{x}}
\newcommand{\By}{\mathbf{y}}    \newcommand{\Bz}{\mathbf{z}}    
\newcommand{\BA}{\mathbf{A}}    \newcommand{\BB}{\mathbf{B}}    \newcommand{\BC}{\mathbf{C}}    \newcommand{\BD}{\mathbf{D}}    \newcommand{\BE}{\mathbf{E}}    \newcommand{\BF}{\mathbf{F}}    \newcommand{\BG}{\mathbf{G}}    \newcommand{\BH}{\mathbf{H}}    \newcommand{\BI}{\mathbf{I}}    \newcommand{\BJ}{\mathbf{J}}    \newcommand{\BK}{\mathbf{K}}    \newcommand{\BL}{\mathbf{L}}
\newcommand{\BM}{\mathbf{M}}    \newcommand{\BN}{\mathbf{N}}    \newcommand{\BO}{\mathbf{O}}    \newcommand{\BP}{\mathbf{P}}    \newcommand{\BQ}{\mathbf{Q}}    \newcommand{\BR}{\mathbf{R}}    \newcommand{\BS}{\mathbf{S}}    \newcommand{\BT}{\mathbf{T}}    \newcommand{\BU}{\mathbf{U}}    \newcommand{\BV}{\mathbf{V}}    \newcommand{\BW}{\mathbf{W}}    \newcommand{\BX}{\mathbf{X}}
\newcommand{\BY}{\mathbf{Y}}    \newcommand{\BZ}{\mathbf{Z}}    

\newcommand{\bfsym}[1]{\ensuremath{\boldsymbol{#1}}}

\def\balpha{\bfsym \alpha}
\def\bbeta{\bfsym \beta}
\def\bgamma{\bfsym \gamma}             \def\bGamma{\bfsym \Gamma}
\def\bdelta{\bfsym {\delta}}           \def\bDelta {\bfsym {\Delta}}
\def\bfeta{\bfsym {\eta}}              \def\bfEta {\bfsym {\Eta}}
\def\bmu{\bfsym {\mu}}                 \def\bMu {\bfsym {\Mu}}
\def\bnu{\bfsym {\nu}}
\def\btheta{\bfsym {\theta}}           \def\bTheta {\bfsym {\Theta}}
\def\beps{\bfsym \varepsilon}          \def\bepsilon{\bfsym \varepsilon}
\def\bsigma{\bfsym \sigma}             \def\bSigma{\bfsym \Sigma}
\def\blambda {\bfsym {\lambda}}        \def\bLambda {\bfsym {\Lambda}}
\def\bomega {\bfsym {\omega}}          \def\bOmega {\bfsym {\Omega}}
\def\brho   {\bfsym {\rho}}
\def\btau{\bfsym {\tau}}
\def\bxi{\bfsym {\xi}}
\def\bzeta{\bfsym {\zeta}}
% May add more in future.
%%%%%%%%%%%%%%%%%%%%%%%%%%%%%%%%%%%%



\theoremstyle{plain}
\newtheorem{theorem}{\quad\quad Theorem}
\newtheorem{proposition}{\quad\quad Proposition}
\newtheorem{corollary}{\quad\quad Corollary}
\newtheorem{lemma}{\quad\quad Lemma}
\newtheorem{example}{Example}
\newtheorem{assumption}{\quad\quad Assumption}
\newtheorem{condition}{\quad\quad Condition}

\theoremstyle{definition}
\newtheorem{remark}{\quad\quad Remark}
\theoremstyle{remark}


%\title{Integrated likelihood ratio test with applications to testing the homogeneity in a two-component normal mixture model}
\title{Integrated Likelihood Ratio Test}

\begin{document}
\maketitle
\begin{abstract}
    A general methodology called integrated likelihood ratio test is proposed which takes posterior Bayes factor and fractional Bayes factor as the main test statistics.
    Different from Bayesian hypothesis testing, we treat the resulting test procedure as frequentist test which controls the significance level.
    Under certain regular conditions, we prove the Wilks phenomenon of the integrated likelihood ratio test and derive its asymptotic local power.
    Our results show that the likelihood ratio test shares similar frequency properties as the likelihood ratio test{\color{red} but have wider application scope}.
    %However, the proposed methodology may exhibit considerable differences from likelihood ratio test.
    Our methodology also includes the statistics produced by approximation computation.
    We apply the proposed method to testing the homogeneity in a two-component normal mixture model.
    This problem is fairly irregular and it is known that the likelihood ratio test statistic has undesirable local power behavior.
    In contrary, the integrated likelihood ratio test has good asymptotic power behavior.
    This shows the superiority of the integrated likelihood ratio test.
\end{abstract}
{\small \textsc{Keywords:} {\em
    Bayes consistency;
   Bayes factor;
   Hypothesis testing.
}}


%\linenumbers

\section{Introduction}

%Suppose we are interested in  testing the hypotheses $H_0:\theta\in \Theta_0$ vs. $H_1:\theta\in \Theta$ for a subset $\Theta_0$ of $\Theta$.


% Where \cite{gelfand1993bayesian} derived the null distribution of PBF.
% However, they didn't explicitly give the conditions needed. In fact, their proof relies on Laplace approximation, which assumes the existence of maximum likelihood estimator (MLE). 
% Note that the existence of MLE implies the existence of LRT. Hence the scope of their method doesn't exceed that of classical LRT\@.

%\cite{Fractional1995} proposed the fractional Bayes factor (FBF).
%The idea of fractional likelihood is also adopted by~\cite{kar10563}.
%We will see that FBF has a wider applicable scope than PBF.

%Both PBF and FBF is a special case of the general ILRT.


%Based on the proof of Bernstein-von Mises theorem (See~\cite{van2000asymptotic} and~\cite{Kleijn2012The}), we give the proof of the Wilks phenomenon and local power of ILRT under fairly weak assumptions.


%Bayesian hypothesis testing is very different from point estimation in that the data can not yanmo prior.



%%%%%% LRT %%%%%%%%%%%%%
%Likelihood ratio test (LRT) is the most widely used statistical testing method which enjoys many optimal properties.
%For example, by Neyman-Pearson lemma, it's the most powerful test in simple null and simple alternative case \citep{Lehmann}.
%In multi-dimensional parameter case, most powerful test does not exist.
%Nevertheless, the LRT is asymptotic optimal in the sense of Bahadur efficiency \citep{MR0315820}.
%However, even in some widely used models, likelihood may be unbounded. See~\cite{Cam1990Maximum} for some examples.
%In this case, LRT does not exist. Another weakness of LRT occurs when the likelihood is not convex in parameters. In this case, numerical algorithms for maximizing likelihood may trap in local maxima. 
%
Likelihood inference plays a dominant role in parametric statistic inference.
On the one hand, the maximum likelihood estimation is asymptotically optimal in a great variety of problems.
On the other hand, the fundamental lemma of Neyman and Pearson tells us that the likelihood ratio test (LRT) is the most powerful test when the null and alternative hypotheses are both simple.
For the general hypothesis testing problem
\begin{equation}\label{eq:newHy}
    H:\theta\in\Theta_0\quad \text{vs.}\quad K: \theta\in \Theta_1,
\end{equation}
where $\Theta_0\cap \Theta_1=\emptyset$, $\Theta_0\cup \Theta_1=\Theta$, $\Theta$ is an open subset of $\mathbb{R}^p$ and $\Theta_0$ is a $p_0$-dimensional subset of $\mathbb{R}^p$,
the LRT statistic is defined as
\begin{equation*}
    \Lambda_{\text{LRT}}=\frac{\max_{\theta\in\Theta}L(\theta)}{\max_{\theta\in\Theta_0} L(\theta)}=\frac{L(\hat{\theta}_{\text{MLE}})}{ L(\hat{\theta}^{(0)}_{\text{MLE}})},
\end{equation*}
where $L(\theta)$ is the likelihood function, $\hat{\theta}_{\text{MLE}}$ and $\hat{\theta}^{(0)}_{\text{MLE}}$ are the MLE of $\theta$ in $\Theta$ and $\Theta_0$, respectively.
A key property of the LRT is Wilks phenomenon~\citep{Wilks1938The} which asserts that for regular models, $2\log \Lambda_{\text{LRT}}$ converges to $\chi^2(p-p_0)$ in law under the null hypothesis.
The LRT has been very successful in many specific problems.
For some moderately complex problems, however, some difficulties may arise when using the LRT.
For example, the maximization of $L(\theta)$ may be difficult if the likelihood function is not concave and has multiple local maxima.
Worse still, in some problems the likelihood is unbounded and hence the LRT is not defined.
See, for example,~\cite{Cam1990Maximum}.
Note that the unbounded likelihood occurs not only in artifical models, but also in some widely used models, for example, the mixture models with unknown component location and scale~\citep{chenjiahua2017}.

In classical goodness of fit test, there are two common types of tests: one based on the maximum (Kolmogorov-Smirnov test, e.g.) and the other based on the integral (Cram\'er-von Mises test, e.g.).
However, the integral type tests draw little attention to frequentist in parametric hypothesis testing problem.
A natural integral type test statistic for hypothesis \eqref{eq:newHy} is defined as
\begin{equation}\label{naiLiS}
\frac{\int_{\Theta}L(\theta) d\Pi(\theta)}{\int_{\Theta_0} L(\theta) d\Pi^{(0)}(\theta)},
\end{equation}
where $\Pi$ and $\Pi^{(0)}$ are some probability measures on $\Theta$ and $\Theta_{0}$, respectively.
If $\Pi(d\theta)$ and $\Pi^{(0)}(d\theta)$ don't rely on data, then~\eqref{naiLiS} has exactly the same form as the Bayes factor with Prior distributions $\Pi$ and $\Pi^{(0)}$.
The Bayes factor, proposed by~\cite{scientificInference}, is the conventional tool for Bayesian hypothesis testing and has been widely used by practitioners (See~\cite{Robert1995Bayes} for a review).
Compared with the methods in other Beyasian inference problem, such as point estimation and credible sets, Bayes factor is developed on its own ground and thus has its own nature.
A notable feature of Bayes factor is that it can not be obtained solely from the posterior distribution of parameters.
There are two consequence of this feature.
First, the computation of Bayes factor is highly nontrivial.
See~\cite{Robert1995Bayes},~\cite{MarkovC},~\cite{raftery2006estimating} and the references therein.
Second, Bayes factor is sensitive to the choice of prior distribution. 
In fact, if the Bayes factor is treated as a frequentist test statistic,
 its asymptotic null distribution relies on the prior density evaluated at the true parameter value.
See, for example,~\cite{clarke1990information}.
Thus, to formulate a frequentist test statistic, the measures $\Pi(d\theta)$ and $\Pi^{(0)}(d\theta)$ in~\eqref{naiLiS} should rely on data.

If $\Pi(\theta=\hat{\theta}_{\text{MLE}})=1$ and $\Pi^{(0)}(\theta=\hat{\theta}^{(0)}_{\text{MLE}})=1$, then~\eqref{naiLiS} equals to the LRT statistic.
In this case, the measure $\Pi$ and $\Pi^{(0)}$ both concentrate on one point which are highly nonsmooth.
For many models, although the LRT fails, the likelihood function $L(\theta)$ still has good properties for most $\theta$ and  the MLE is trapped in a fairly small area of $\theta$ where $L(\theta)$ has bad behavior.
{\color{red} A figure?}
In such cases, some smoother $\Pi$ and $\Pi^{(0)}$ may perform better intuitively.
Following this idea, a natural choice is to take $\Pi$ and $\Pi^{(0)}$ as the posterior distribution (with respect to certain predefined prior distribution) of $\theta$ in $\Theta$ and $\Theta_0$, respectively.
In this case, the statistic~\eqref{naiLiS} becomes the posterior Bayes factor proposed (PBF) by~\cite{Aitkin1991Posterior}.
~\cite{Aitkin1991Posterior} argued that if the likelihood is concentrated around the MLE, the PBF should approximately equal to $2^{(p-p_0)}\Lambda_{\text{LRT}}$.
This implies that PBF has a similar Wilks phenomenon as the LRT.
%Note that if the likelihood is concentrated around the MLE, then the MLE is consistent and thus the LRT can be applied.
%Hence we are more interested in the case where the MLE is inconsistent.
%In a theoretical point of view, the Wilks phenomenon of PBF requires certain conditions which are undesirable.
%In contrast, it is well known that the posterior distribution tends to become independent of the prior distribution as the sample size increases.
%Several modifications of Bayes factor have been proposed.



Note that for any $a>0$, the LRT is equivalent to
\begin{equation*}
    \frac{\max_{\theta\in\Theta}L^{a}(\theta)}{\max_{\theta\in\Theta_0}L^a(\theta)}.
\end{equation*}
In the contrary, the statistic
\begin{equation}\label{naiLiS2}
\frac{\int_{\Theta}L^a(\theta) d\Pi(\theta)}{\int_{\Theta_0} L^a(\theta) d\Pi^{(0)}(\theta)}
\end{equation}
is not equivalent to~\eqref{naiLiS}.
We shall consider the test statistic~\eqref{naiLiS2} with $0<a<1$.
Correspondingly, the measure $\Pi$ and $\Pi^{(0)}$ can also take the fractional posterior~\cite{Bha2016}.
Raising the likelihood to a fractional power has several advantages. See, for example,~\cite{kar10563} and~\cite{Bha2016}.
In particular, the consistency of the fractional posterior requires less conditions than the consistency of the usual posterior.
A special case of~\eqref{naiLiS} is the fractional Bayes factor (FBF) proposed by \cite{Fractional1995}.
We call~\eqref{naiLiS} the generalized FBF if $\Pi$ and $\Pi^{(0)}$ are fractional posterior distributions.

Under certain regular conditions, we rigorously prove the Wilks phenomenon of the generalized FBF.
Based on the Wilks phenomenon, an asymptotically correct frequentist test procedure can be formulated.
We also give the asymptotic power of the resulting test procedure under contiguous alternative.
It is shown that the generalized FBF has similar asymptotic behavior as the LRT.
{\color{red}However, it has wider application scope.}


%The frequentist properties of Bayesian methods have drawn much attention in recent years.
%See~\cite{ghosal2000},~\cite{Shen2001Rates},~\cite{vaart2007convergence},~\cite{Kleijn2012The} and the references therein.
%These works show that many Bayesian methods still perform well when they are treated as frequentist methods.
%%However, existing research is largely concerned with the consistency and asymptotic normality of the posterior distribution.
%Existing research is largely concerned with the frequentist properties of Bayesian point estimation and credible sets.
%For these problems, Bayesian methods can be directly treated as frequentist methods.
%However, for testing problem, Bayesian methods are not required to control the type I error rate and hence can not be directly treated as frequentist methods.

%Compared with likelihood ratio test which utilize the maximum of the likelihood, Bayesian methods integrate the likelihood by a weight function.
%Can these Bayesian methods be formulated into frequentist tests?
%If they can, what's the behavior of these tests?
%This paper is devoted to answering such questions.

%Motivated by this, we propose a flexible methodology called integrated likelihood ratio test (ILRT) which takes PBF and FBF as special examples.
%ILRT also includes methods that are produced by approximation computation.



The generalized FBF can be computed by sampling $\theta$ from the fractional posterior and calculate the sample mean of the fractional likelihood.
For moderately complex model, however, sampling from the fractional posterior may be difficult and hence some approximation methods may be used in practice.
%The computations of PBF and FBF are easier than that of Bayes factor since the PBF and FBF can be computed by sampling the likelihood according to posterior distribution or fractional posterior distribution.
Variational inference is a popular method for approximating intractable posterior distribution.
See~\cite{blei2017} and the references therein.
Such procedure produce a distribution other than the fractional posterior distribution.
To accommodate such cases, we also give a theorem for the general measure $\Pi$ and $\Pi^{(0)}$ in~\eqref{naiLiS2}.
{\color{red} Allow $a+b=1$.}





For some irregular problems, the behavior of likelihood is complicated.
%Since the integral of the likelihood can smooth the irregular behavior of the likelihood, it can be expected that ILRT may have better behavior than the LRT.
%To illustrate this point,
We apply the proposed method to testing the homogeneity in a two-component normal mixture model.
This problem is fairly irregular and suffers from nonidentifiability and nonconvex likelihood.
~\cite{HALL2005158} showed that the likelihood ratio test has trivial power under $n^{-1/2}$ local alternative hypothesis. 
In contrary, we show that the ILRT have nontrivial power under $n^{-1/2}$ local alternative hypothesis.



The paper is organized as follow.
In Section 2, we prove the Wilks phenomenon of the ILRT statistic and gives the asympototic local power.
In Section 3, we apply ILRT to testing the homogeneity in a two-component normal mixture model.
Section 4 concludes the paper.
All technical proves are in Appendix.





\section{Integrated likelihood ratio test}

\subsection{The test statistic}

Let $\BX^{(n)}=(X_1,\ldots,X_n)$ be independent identically distributed (iid) observations with values in some space $(\mathcal{X};\mathscr{A})$.
Suppose that there is a $\sigma$-finite measure $\mu$ on $\mathcal{X}$ and that the  possible distribution $P_\theta$ of $X_i$ has a density $p(X|\theta)$ with respect to $\mu$.
Denote by $P_{\theta}^{n}$ the joint distribution of $\BX^{(n)}$.
Let $p_{n}(\BX^{(n)}|\theta)=\prod_{i=1}^n p(X_i|\theta)$ denote the density of $P_{\theta}^n$ with respect to the $n$-fold product measure $\mu^n$.
The parameter $\theta$ takes its values in $\Theta$, a subset of $\mathbb{R}^{p}$.
Suppose $\theta=(\nu^T,\xi^T)^T$, where $\nu$ is a $p_0$ dimensional subvector and $\xi$ is a $p-p_0$ dimensional subvector.
 We would like to test the hypotheses
\begin{equation*}
    H:\theta\in\Theta_0\quad \text{v.s.}\quad K:\theta\in\Theta\backslash \Theta_0,
\end{equation*}
where the null space $\Theta_0$ is a $p_0$-dimensional subspace of $\Theta$ defined as
\begin{equation*}
    \Theta_0=\{(\nu^T,\xi^T)^T:(\nu^T,\xi^T)^T\in\Theta, \, \xi=\xi_0\}.
\end{equation*}
If the null hypothesis is true, we denote by $\theta_0=(\nu_0^T,\xi_0^T)^T$ the true parameter which generates the data.



In Bayesian hypothesis testing framework, one puts prior $\pi(\nu)$ and $\pi(\theta)$ on parameters under the null and alternative hypotheses, respectively.
The conventional Bayes factor is defined as
\begin{equation*}
  \frac{\int_{\Theta} p_n(\BX^{(n)}|\theta)\pi(\theta)\, d\theta}
    {\int_{\tilde{\Theta}_0}p_n(\BX^{(n)}|\nu,\xi_0)\pi(\nu)\, d\nu},
\end{equation*}
where $\tilde{\Theta}_0=\{\nu: (\nu^T,\xi^T)^T\in \Theta_0\}$.
However, Bayes factor is sensitive to the specification of prior, which may cause difficulties in the absense of a well-formulated subjective prior. See, for example, \cite{Lindley1982}.
%The frequency property of Bayes factor is not satisfactory.
%To overcome this weakness, several robust alternatives to Bayes factor have been proposed.
To deal with this problem,~\cite{Aitkin1991Posterior} proposed PBF which is defined to be
\begin{equation*}
    \frac{\int_{\Theta} p_n(\BX^{(n)}|\theta)\pi(\theta|\BX^{(n)})\, d\theta}{\int_{\tilde{\Theta}_0}p_n(\BX^{(n)}|\nu,\xi_0)\pi(\nu|\BX^{(n)})\, d\nu},
\end{equation*}
where $\pi(\nu|\BX^{(n)})$ and $\pi(\theta|\BX^{(n)})$ are the posterior densities under the null and alternative hypothesis, respectively.
\cite{Fractional1995} proposed  FBF which is defined as
\begin{equation*}
    \frac{L_{1}}{L_{b}}\cdot \frac{L_{b}^{(0)}}{L_{1}^{(0)}}\quad \text{for}\quad 0<b<1,
\end{equation*}
where for $t>0$,
 $$
 L_t=\int_{\Theta}\big[p_n(\BX^{(n)}|\theta)\big]^t \pi(\theta)\, d\theta,\quad
 L_t^{(0)}=\int_{\Theta_0}\big[p_n(\BX^{(n)}|\nu,\xi_0)\big]^t \pi(\nu)\, d\nu.
 $$

We generalize the PBF and FBF and propose the ILRT statistic as
\begin{equation*}
    \frac{\int_{\Theta} \big[p_n(\BX^{(n)}|\theta)\big]^{a}\pi(\theta;\BX^{(n)})\,d\theta}{\int_{\tilde{\Theta}_0} \big[p_n(\BX^{(n)}|\nu,\xi_0)\big]^{a}\pi(\nu;\BX^{(n)})\,d\nu},
\end{equation*}
where $a>0$ is a hyperparameter,
 the weight functions $\pi(\theta;\BX^{(n)})$ and $\pi(\nu;\BX^{(n)})$ are probability density functions in $\Theta$ and $\tilde{\Theta}_0$ respectively.
Note that $\pi(\theta;\BX^{(n)})$ and $\pi(\nu;\BX^{(n)})$ may be data dependent but does not need to be the posterior density.
If we take the weight function as
\begin{equation}\label{firstWeight}
\pi(\theta;\BX^{(n)})=\frac{\big[p_n(\BX^{(n)}|\theta)\big]^b \pi(\theta)}{\int_{\Theta}\big[p_n(\BX^{(n)}|\theta)\big]^b \pi(\theta)\, d\theta},
\quad
\pi(\nu;\BX^{(n)})=\frac{\big[p_n(\BX^{(n)}|\nu,
    \xi_0)\big]^b \pi(\nu)}{\int_{\tilde{\Theta}_0}\big[p_n(\BX^{(n)}|\nu,\xi_0)\big]^b \pi(\nu)\, d\nu},
\end{equation}
then the ILRT statistic equals to
$$
    \Lambda_{a,b}=
    \frac{L_{a+b}}{L_{b}}\cdot \frac{L_{b}^{(0)}}{L_{a+b}^{(0)}}.
$$
We shall call $\Lambda_{a,b}$ the generalized FBF throughout the paper.
The FBF and PBF are both the special cases of the generalized FBF.
In fact, the FBF equals to $\Lambda_{1,b}$, the PBF equals to $\Lambda_{2,1}$.

The computation of the ILRT statistic is relatively simple.
We can independently generate $\theta_1,\ldots,\theta_m$ and $\nu_1,\ldots,\nu_m$ according to $\pi(\theta;\BX^{(n)})$ and $\pi(\nu;\BX^{(n)})$ for a large $m$.
Then the ILRT statistic can be approximated by
\begin{equation*}
    \frac{\sum_{i=1}^m \big[p_n(\BX^{(n)}|\theta_i)\big]^{a}}{\sum_{i=1}^m\big[p_n(\BX^{(n)}|\nu_i,\xi_0)\big]^{a}}.
\end{equation*}


For some moderately complex models,~\eqref{firstWeight} may be complicated.
Consequently, sampling from~\eqref{firstWeight} may be intractable.
In this case, one may use some simple form weight function to approximate~\eqref{firstWeight}.
A popular method for approximating~\eqref{firstWeight} is variational inference.
See. See, for example,~\cite{blei2017}.
In this case, the weight function in~\eqref{eq:definition} is equals to the variational apprixomation of~\eqref{firstWeight}.
The ILRT methodology also includes such approximate method.




\subsection{Generalized FBF}
In this section, we investigate the asymptotic behavior of the generalized FBF.
The following assumption is adapted from~\cite{Kleijn2012The} and is satisfied by many common models. 
%The function $\theta \mapsto \log p(X|\theta)$ is differentiable at $\theta_0$  $P_{\theta_0}$-a.s.\ with derivative 
%$$\dot{\ell}_{\theta_0}(X)=\frac{\partial}{\partial \theta}\log p(X|\theta)\Big|_{\theta=\theta_0}.$$
%There's an open neighborhood $V$ of $\theta_0$ such that for every $\theta_1,\theta_2\in V$,
        %\begin{equation*}
            %|\log p(X|\theta_1)-\log p(X|\theta_2)|\leq m(X)\|\theta_1-\theta_2\|,
        %\end{equation*}
        %where $m(X)$ is a measurable function satisfying $P_{0}\exp[s m(X)]<\infty$ for some $s>0$.
        %The Fisher information matrix $I_{\theta_0}=P_{\theta_0}\dot{\ell}_{\theta_0}\dot{\ell}_{\theta_0}^T$ is positive-definite and as $\theta\to \theta_0$,
    %\begin{equation*}
        %P_{\theta_0} \log \frac{p(X|\theta)}{ p(X|\theta_0)}
        %=-\frac{1}{2}(\theta-\theta_0)^T I_{\theta_0} (\theta-\theta_0)+o(\|\theta-\theta_0\|^2).
    %\end{equation*}
\begin{assumption}\label{Assumption1}
    The parameter space $\Theta$ and $\tilde{\Theta}_0$ are open subsets of $\mathbb{R}^p$ and $\mathbb{R}^{p_0}$, respectively.
    The parameter $\theta_0$ and $\nu_0$ are inner points of $\Theta$ and $\tilde{\Theta}_0$, respectively.
    The derivative 
$$\dot{\ell}_{\theta_0}(X)=\frac{\partial}{\partial \theta}\log p(X|\theta)\Big|_{\theta=\theta_0}$$
exists $P_{\theta_0}$-a.s.\ and satisfies $P_{\theta_0}\dot{\ell}_{\theta_0}=0_p$.
The Fisher information matrix $I_{\theta_0}=P_{\theta_0}\dot{\ell}_{\theta_0}\dot{\ell}_{\theta_0}^T$ is positive-definite.
For every $M>0$,
    \begin{equation*}
        \sup_{\|h\|\leq M}\Big|
         \log \frac{p_n(\BX^{(n)}|\theta_0+n^{-1/2}h)}{p_n(\BX^{(n)}|\theta_0)}-h^T I_{\theta_0}\Delta_{n,\theta_0}+\frac{1}{2}h^T I_{\theta_0}h
         \Big|\xrightarrow{P^n_{\theta_0}}0,
    \end{equation*}
where $\Delta_{n,\theta_0}=\sum_{i=1}^n I_{\theta_0}^{-1}\dot{\ell}_{\theta_0}(X_i)$.
\end{assumption}     
%, it ensures a local asymptotic normality expansion of likelihood. See Lemma~\ref{Thm:localExpansion} in Appendix.
    For $t>0$, we say $L_t$ is $\sqrt{n}$-consistent if for every $M_n\to \infty$,
    $$
    \frac{L_t({\{\theta:\|\theta-\theta_0\|> M_n/\sqrt{n}\}})}{L_t}\xrightarrow{P_{\theta_0}^n} 0,
    $$
    where for a set $A\subset \Theta$, $L_t (A)=\int_{A} [ {p_n(\BX^{(n)}|\theta)} ]^{t} \pi(\theta) \, d \theta$.
    The $\sqrt{n}$-consistency of $L_t^{(0)}$ is similarly defined.
    Note that the consistency of $L_1$ is equivalent to the consistency of the posterior distribution.
    In~\cite{Kleijn2012The}, the $\sqrt{n}$-consistency of posterior distribution is a key assumption to prove Bernstein-von Mises theorem.
    Likewise, the $\sqrt{n}$-consistency of $L_t$ is a key assumption of the following theorem.



    \begin{theorem}\label{Thm:maintheorem}
        Suppose that Assumption~\ref{Assumption1} holds, $L_{a+b}$, $L_b$, $L_{a+b}^{(0)}$ and $L_b^{(0)}$ are $\sqrt{n}$-consistent, $\pi(\theta)$ is continuous at $\theta_0$ with $\pi(\theta_0)>0$, $\pi(\nu)$ is continuous at $\nu_0$ with $\pi(\nu_0)>0$. Then for $\{\theta_n\}$ such that $\sqrt{n}(\theta_n-\theta_0)\to \eta$, 
        $$
        2\log \Lambda_{a,b}\overset{P^n_{\theta_n}}{\rightsquigarrow}-{(p-p_0)}\log (1+\frac{a}{b})+{a}\chi^2(p-p_0,\delta),
        $$
        where $\chi^2(p-p_0,\delta)$ is a noncentral chi-squared random variable with $p-p_0$ degrees of freedom and noncentrality parameter $\delta=\eta^T\big( I_{\theta_0}-I_{\theta_0} J(J^T I_{\theta_0} J)^{-1}J^T I_{\theta_0}\big)\eta$ and $J=(I_{p_0},0_{p_0\times(p-p_0)})^T$,
``$\rightsquigarrow$'' means weak convergence.
    \end{theorem}
Theorem~\ref{Thm:maintheorem} gives the asymptotic distribution of $2\log \Lambda_{a,b}$ under the null hypothesis and the local alternative hypothesis.
To obtain a test with asymptotic type I error rate $\alpha$, the critical value of $2\log \Lambda_{a,b}$ can be defined to be $-(p-p_0)\log (1+a/b)+ a\chi^2_{1-\alpha}(p-p_0)$, where $\chi^2_{1-\alpha}(p-p_0)$ is the $1-\alpha$ quantile of a chi-squared random variable with $p-p_0$ degrees of freedom.
By Theorem~\ref{Thm:maintheorem}, the resulting test has local asymptotic power
\begin{equation}\label{eq:likelihoodPower}
\Pr \left( \chi^2(p-p_0,\delta)> \chi^2_{1-\alpha}(p-p_0) \right).
\end{equation}
It is known that, under certain regular conditions,~\eqref{eq:likelihoodPower} is also the local asymptotic power of the likelihood ratio test. 
In this view, $\Lambda_{a,b}$ enjoys good frequentist properties.


 The $\sqrt{n}$-consistency of $L_t$ is a key assumption of Theorem \ref{Thm:maintheorem}.
Hence we would like to give sufficient conditions for the $\sqrt{n}$-consistency of $L_t$.
 First we consider the exponential family.
\begin{proposition}\label{exponentialCon}
    Suppose $p(X|\theta)=\exp\big[\theta^T T(X)-A(\theta)\big]$, $\Theta$ is an open subset of $\mathbb{R}^p$, $\theta_0$ is an interior point of $\Theta$, 
    $$I_{\theta_0}=\frac{\partial^2}{\partial \theta \partial \theta^T} A(\theta_0)>0.$$
    Then $L_{t}$ is consistent for $t>0$.
\end{proposition}

Proposition~\ref{exponentialCon} establishes the $\sqrt{n}$-consistent of $L_t$ for all $t>0$ under full-rank exponential family models.
If the full model and the null model both belong to the full-rank exponential family, Assumption~\ref{Assumption1} is also satisfied.
In this case, Theorem~\ref{Thm:maintheorem} implies that the generalized FBF can be used as frequentist test.
However,
for any test methodology, the success in the full-rank exponential family models is just a minimal requirement 
since for these models, LRT is also easy to implement and enjoys good asymptotic properties.
We would like to consider more general models.


For general models, the likelihood function may not be convex.
This often makes it hard to implement the LRT.
For some models, a more serious problem may occur, that is, the likelihood may be unbounded and hence the LRT can not be defined.
This problem may occur even if the likelihood function has good local analytical properties, such as location-scale mixture models.  See \cite{Cam1990Maximum} for more examples.
A natural question is that if the fractional integrated likelihood $L_t$ is always well defined.
The following theorem shows that $L_t$ is always well defined for $t\leq 1$ and is not well defined for some model for $t>1$.
%For more general models, however, the $\sqrt{n}$-consistency of $L_t$ needs further conditions.
%For more general models, however, the $\sqrt{n}$-consistency of $L_t$ is more complicated.
\begin{proposition}
    If $t\leq 1$, $L_t< +\infty$ $P_{\theta_0}^n$-a.s. for any models. If $t> 1$, $L_t = +\infty$ for some models.
    \label{exprop}
\end{proposition}


%It turns out that the behavior of $L_t$ for $t>1$ and $t\leq 1$ are different.

Because of the bad behavior of $L_t$ for $t>1$, next we only consider $L_t$ for $t\leq 1$.
 For $t=1$, the $\sqrt{n}$-consistency of $L_t$ is equivalent to the $\sqrt{n}$-consistency of the posterior distribution.
 The consistency of posterior distribution has drawn much attention in the literature.
 See, for example,~\cite{ghosal2000},~\cite{Shen2001Rates},~\cite{vaart2007convergence}.
A popular and convenient way of establishing the consistency of posterior is through the condition that suitable test sequences exist.
This approach is adopted by~\cite{ghosal2000},~\cite{vaart2007convergence} and~\cite{Kleijn2012The}.

\begin{assumption}\label{Assumption2}
    For every $\epsilon>0$, there exists a sequence of tests $\phi_n$ such that
        \begin{equation*}
            P_{\theta_0}^n\phi_n\to 0,\quad \sup_{\|\theta-\theta_0\|\geq \epsilon} P_\theta^n(1-\phi_n)\to 0.
        \end{equation*}
\end{assumption}
Assumption~\ref{Assumption2} is satisfied when the parameter space is compact and the model is suitably continuous. See Theorem 3.2 of~\cite{Kleijn2012The}.

\begin{proposition}[\cite{Kleijn2012The}, Theorem 3.1]
    Suppose $\theta_0$ is an interior of $\Theta$, $\pi(\theta)$ is continuous at $\theta_0$ and $\pi(\theta_0)>0$.
    Under Assumptions \ref{Assumption1} and~\ref{Assumption2}, $L_1$ is consistent.
\end{proposition}


%The consistency of $L_t$ for $t>1$ can be proved under conditions similar to Assumption~\ref{Assumption2}.
%However, the requirement on the sequence $\{\phi_n\}$ lacks statistical interpretation for $t>1$.



The consistency of $L_t$ for $0<t<1$ is different from $t=1$.
\cite{kar10563} considered the Hellinger consistency of $L_{1/2}$.
It is shown that the consistency of $L_{1/2}$ does not need Assumption~\ref{Assumption2}.
However, they only consider $t=1/2$ and didn't consider the $\sqrt{n}$-convergence result.
Recently,~\cite{Bha2016} further developed the idea of~\cite{kar10563} and derived a general bounds for the consistency of $L_t$ for $0<t<1$.
However, their result can not yield the $\sqrt{n}$-consistency for parametric models.
We shall prove the $\sqrt{n}$-consisency of $L_{t}$ for $0<t<1$ under certain conditions on the R\'{e}nyi divergence between distributions in the family $\{P_\theta:\theta\in\Theta\}$.

 For two parameters $\theta_1$ and $\theta_2$, the $\alpha$ order R\'{e}nyi divergence ($0<\alpha<1$) of $P_{\theta_1}$ from $P_{\theta_2}$ is defined to be
$$
D_{\alpha}(\theta_1||\theta_2)=-\frac{1}{1-\alpha}\log \rho_{\alpha}(\theta_1,\theta_2),
$$
where
$
\rho_{\alpha}(\theta_1,\theta_2)=\int_{\mathcal{X}} p(X|\theta_1)^{\alpha} p(X|\theta_2)^{1-\alpha} \, d \mu
$ is the so-called Hellinger integral.
The following assumption is needed for our $\sqrt{n}$-consistency result.
\begin{assumption}\label{Assumption4}
    For some $\alpha\in(0,1)$, there exist positive constancts $\delta$, $\epsilon$ and $C$ such that,
     $D_{\alpha}(\theta||\theta_0)  \geq  C \|\theta-\theta_0\|^2$ for $\|\theta-\theta_0\|\leq \delta$ and $D_{\alpha}(\theta||\theta_0) \geq \epsilon$ for $\|\theta-\theta_0\|>\delta$.
\end{assumption}
\begin{remark}
    A remarkable property of R\'{e}nyi divergence is the equivalence of all $D_{\alpha}$: If $0<\alpha<\beta<1$, then
    $$
    \frac{\alpha}{1-\alpha}\frac{1-\beta}{\beta} D_{\beta}(\theta_1||\theta_2)
    \leq D_{\alpha}(\theta_1||\theta_2)\leq D_{\beta}(\theta_1||\theta_2).
    $$
    See, for example,~\cite{2016arXiv160801805B}.
    As a result, if Assumption~\ref{Assumption4} holds for some $\alpha\in(0,1)$, then it will hold for every $\alpha\in(0,1)$.
\end{remark}
To appreciate Assumption~\ref{Assumption4},
   suppose, for example, that $D_{\alpha}(\theta||\theta_0)$ is twice continuously differentiable in $\theta$.
   Since $\theta=\theta_0$ is a minumum point of  $D_{\alpha}(\theta||\theta_0)$, the first order derivative of $D_{\alpha}(\theta||\theta)$ at $\theta=\theta_0$ is zero and the second order derivative at $\theta=\theta_0$ is positive semidefinite.
By Taylor theorem, in a small neighbourhood of $\theta_0$,
   $$
   D_{\alpha}(\theta||\theta_0)=\frac{1}{2}(\theta-\theta_0)^T \frac{\partial^2}{\partial \theta \partial \theta^T} D_{\alpha}(\theta||\theta_0)\Big|_{\theta=\theta^*}  (\theta-\theta_0),
   $$
   where $\theta^*$ is between $\theta_0$ and $\theta$.
   If we further assume the second order derivative is positive definite at $\theta=\theta_0$, then in a small neighbourhood of $\theta_0$, there is a positive constant $C$ such that $D_{\alpha}(\theta||\theta_0)\geq C\|\theta-\theta_0\|^2$.
   Thus, Assumption~\ref{Assumption4} is a fairly weak condition.
%%%%%%%%%%%%%%% Here we haven't consider the second condition. It can be done through a minimax lower bound argument.
\begin{proposition}\label{Theoremless1}
    Suppose $\theta_0$ is an interior of $\Theta$, $\pi(\theta)$ is continuous at $\theta_0$ and $\pi(\theta_0)>0$.
    Under Assumptions \ref{Assumption1} and~\ref{Assumption4}, for fixed $t\in(0,1)$, $L_t$ is consistent.
\end{proposition}

Compared with Assumption~\ref{Assumption2}, it appears that Assumption~\ref{Assumption4} is easier to verify.
Note that the asymptotic power of $\Lambda_{a,b}$ is independent of $a,b$.
Hence it can be recommended to use the generalized FBF with $a+b< 1$.


\subsection{General weight function}

For some moderately complex models, the fractional posterior~\eqref{firstWeight} are not easy to calculate.
In this case, we can use simpler weight functions to approximate~\eqref{firstWeight}.

Let $h=\sqrt{n}(\theta-\theta_0)$ be the local parameter.
Then the posterior density of $h$ is $\pi_n(h|\BX^{(n)})=n^{-1/2}\pi_n(\theta|\BX^{(n)})$.
Theorem 2.1 of \cite{Kleijn2012The} states that under Assumptions~\ref{Assumption1} and~\ref{Assumption2},
$$
            \|\pi_n(h|\BX^{(n)})-\phi(h;\Delta_{n,\theta_0},I_{\theta_0}^{-1})\|\overset{P_{\theta_0}^n}{\to}0,
$$
where $\Delta_{n,\theta_0}=n^{-1/2}\sum_{i=1}^n I_{\theta_0}^{-1}\dot{\ell}_{\theta_0}(X_i)$ and for two densities $q_1(h)$ and $q_2(h)$, $\|q_1(h)-q_2(h)\|=\int |q_1(h)-q_2(h)|\, dh$ is their total variation distance.
We shall assume that the weight function inherits this property.
        
\begin{assumption}\label{Assumption3}
    Let $b\in(0,1)$ be a fixed number.
    Assume that $\pi_n(h;\BX^{(n)})$ satisfies
        \begin{equation}\label{vonMisesResults}
            \|\pi_n(h;\BX^{(n)})-\phi(h;\Delta_{n,\theta_0},b^{-1}I_{\theta_0}^{-1})\|\overset{P_{\theta_0}^n}{\to}0.
        \end{equation}

        Similarly, let $h^{(0)}=\sqrt{n}(\nu-\nu_0)$. Define $\pi_n(h^{(0)};\BX^{(n)})=n^{-1/2}\pi_n(\nu;\BX^{(n)})$. Assume that 
\begin{equation}\label{vonMisesResultsl}
    \|\pi_n(h^{(0)};\BX^{(n)})-\phi(h^{(0)};\Delta^{(0)}_{n,\theta_0},b^{-1}I_{\theta_0}^{(0)-1})\|\overset{P_{\theta_0}^n}{\to}0,
\end{equation}
where $$ \Delta_{n,\theta_0}^{(0)}
=\frac{1}{\sqrt{n}}\sum_{i=1}^n I_{\theta_0}^{(0)-1}\dot{\ell}^{(0)}_{\theta_0}(X_i),
\quad I^{(0)}_{\theta_0}=P_{\theta_0}\dot{\ell}_{\theta_0}^{(0)}\dot{\ell}_{\theta_0}^{(0)T},\quad 
\dot{\ell}^{(0)}(X)=\frac{\partial}{\partial \nu}\log p(X|\nu,\xi_0)\Big|_{\nu=\nu_0}
.
$$
Furthermore, assume that for every $\epsilon>0$, there exists Lebesgue integrable functions $T(h)$ and $T^{(0)}(h)$ such that 

    \begin{equation}\label{Assump31}
        \lim_{n\to \infty}P_{\theta_0}^n\left\{\sup_{h\in \mathbb{R}^p}(\pi_n(h;\BX^{(n)})-T(h))\leq 0\right\}\geq 1-\epsilon.
\end{equation}
    \begin{equation}\label{Assump31l}
        \lim_{n\to \infty}P_{\theta_0}^n\left\{\sup_{h^{(0)}\in \mathbb{R}^{p_0}}(\pi_n(h^{(0)};\BX^{(n)})-T^{(0)}(h^{(0)}))\leq 0\right\}\geq 1-\epsilon.
\end{equation}


\end{assumption}
The conditions~\eqref{Assump31} and~\eqref{Assump31l} in Assumption~\ref{Assumption3} assume that there is a function controlling the tail of the weight functions.
We need to control the tail of the weight function since the behavior of the likelihood may be undersirable when $\theta$ is far away from $\theta_0$.
In fact, even for some fairly regular models, the likelihood may tends to infinity, which invalidates LRT.
See, for example,~\cite{Cam1990Maximum}.
So we control the tail of the weight function to avoid too much weights on the tail of likelihood.
If the weight function $\pi_n(h;\BX^{(n)})$ is normal density, then it can be shown that the conditions~\eqref{vonMisesResults} and~\eqref{vonMisesResultsl} implies~\eqref{Assump31} and~\eqref{Assump31l}.


Under Assumption~\ref{Assumption3}, we consider the ILRT statistic
\begin{equation}
    \Lambda^*_{a,b}=\frac{\int_{\Theta} \big[p_n(\BX^{(n)}|\theta)\big]^{a}\pi(\theta;\BX^{(n)})\,d\theta}{\int_{\tilde{\Theta}_0} \big[p_n(\BX^{(n)}|\nu,\xi_0)\big]^{a}\pi(\nu;\BX^{(n)})\,d\nu}.
\label{eq:definition}
\end{equation}
The following theorem gives the asymptotic distribution of ILRT statistic.


\begin{theorem}\label{theoremMain}
    Suppose that Assumptions~\ref{Assumption1} and \ref{Assumption3} hold, the true parameter $\theta_0$ is an interior point of $\Theta$, $\nu$ is a relative interior point of $\tilde{\Theta}_0$.
    Suppose $a+b\leq 1$.
    Then for $\{\theta_n\}$ such that $\sqrt{n}(\theta_n-\theta_0)\to \eta$, we have

        $$
        2\log \Lambda^*_{a,b}\overset{P^n_{\theta_n}}{\rightsquigarrow}-{(p-p_0)}\log (1+\frac{a}{b})+{a}\chi^2(p-p_0,\delta),
        $$
        where $\delta$ is defined as in Theorem~\ref{Thm:maintheorem}.
\end{theorem}
Theorem~\ref{theoremMain} shows that even with approximate weight function, the ILRT statistic can still produce an asymptotic optimal test.
A practical method to obtain simple form weight function $\pi_n(h;\BX^{(n)})$ is the variational inference. See, for example,~\cite{blei2017}.
Next we shall consider a simple variational method which is guaranteed to yield a weight function satisfying Assumption~\ref{Assumption3}.
For comprehensive considerations of the statistical properties of variational methods, see the recent works of~\cite{yixin2017},~\cite{pati2017} and~\cite{yunyang2017}.

%The following example shows that the weight function obtained from R\'{e}nyi divergence variational inference satisfies Assumption~\ref{Assumption3}.

Let $\mathcal{Q}$ be the family of all $p$ dimensional normal distribution.
Let $\pi_n(\theta;\BX^{(n)})$ be the fractional posterior of order $b$ and $\pi_n(h;\BX^{(n)})=n^{-1/2}\pi_n(\theta_0+n^{-1/2}h;\BX^{(n)})$ be the corresponding fractional posterior of $h$.
Suppose that $\pi_n(h;\BX^{(n)})$ satisfies Bernstein-von Mises theorem,
\begin{equation}\label{eq:xiebuwanlaaa}
    \|\pi_n(h;\BX^{(n)})-\phi(h;\Delta_{n,\theta_0},b^{-1}I_{\theta_0}^{-1})\| \xrightarrow{P_{\theta_0}^n}0.
\end{equation}
Let the weight function $\pi^{\dagger}_n(\theta;\BX^{(n)})$ be the normal approximation of $\pi_n(h;\BX^{(n)})$ obtained from R\'{e}nyi divergence variational inference~\citep{NIPS2016_6208}, that is,
    $$
    \pi_n^{\dagger}(\theta;\BX^{(n)})=\argmin_{q(\theta)\in\mathcal{Q}} -\frac{1}{1-\alpha} \log \int q(\theta)^{\alpha} \pi_n(\theta;\BX^{(n)})^{1-\alpha}\, d\theta,
    $$
    where $0<\alpha<1$ is an arbitrary constant.
    Let $\pi_n^{\dagger}(h;\BX^{(n)})=n^{-1/2}\pi_n^{\dagger}(\theta_0+n^{-1/2}h;\BX^{(n)})$ be the weight function of $h$.
    It can be seen that
    $$
    \pi_n^{\dagger}(h;\BX^{(n)})=\argmin_{q(h)\in\mathcal{Q}} -\frac{1}{1-\alpha} \log \int q(h)^{\alpha} \pi_n(h;\BX^{(n)})^{1-\alpha}\, dh.
    $$
    Hence we have
    \begin{equation}\label{eq:xiebuwan}
        -\frac{1}{1-\alpha} \log \int \pi^{\dagger}(h;\BX^{(n)})^{\alpha} \pi_n(h;\BX^{(n)})^{1-\alpha}\, dh
    \leq
    -\frac{1}{1-\alpha} \log \int \phi(h;\Delta_{n,\theta_0}, I_{\theta_0}^{-1})^{\alpha} \pi_n(h|\BX^{(n)})^{1-\alpha}\, dh.
    \end{equation}
    Since R\'{e}nyi divergence and total variation distance are equivalent,~\eqref{eq:xiebuwanlaaa} implies that the right hand side of~\eqref{eq:xiebuwan} tends to $0$ in $P_{\theta_0}^n$-probability.
    Again by the equivalence of R\'{e}nyi divegence and total variation distance, we have
\begin{equation*}
    \|\pi_n^{\dagger}(h;\BX^{(n)})-\phi(h;\Delta_{n,\theta_0},b^{-1}I_{\theta_0}^{-1})\| \xrightarrow{P_{\theta_0}^n}0.
\end{equation*}
    Since $\pi_n(h;\BX^{(n)})$ is a normal density,~\eqref{vonMisesResults} implies the mean and covariance parameter of $\pi_n(h;\BX^{(n)})$ converges to $\Delta_{n,\theta_0}$ and $I_{\theta_0}^{-1}$ respectively.
    Then a dominating function $T(h)$ exists and thus~\eqref{Assump31} holds.

\section{Normal mixture model}
In this section, we apply the ILRT methodology to the testing the component number of normal mixture model.
Normal mixture model is a highly irregular model.
Due to partial loss of identifiability, the likelihood ratio test has undesirable behavior.
For example, if the component variances are totally unknown, the likelihood is bounded and thus likelihood ratio test is not defined \citep{Cam1990Maximum}.
See~\cite{chenjiahua2017} for a review of the testing problems for mixture models.
Since the integral of the likelihood can smooth the irregular behavior of the likelihood, it can be expected that ILRT may have better behavior than likelihood ratio test.
For example, for unknown variances case, ILRT is at least well defined.

Suppose $X_1,\ldots,X_n$ are iid distributed as a mixture of normal distributions
\begin{equation*}
    p(X|\omega,\xi,\sigma^2)=\frac{1-\omega}{\sqrt{2\pi}}\exp\big(-\frac{1}{2}X^2\big)
+\frac{\omega}{\sqrt{2\pi}\sigma}\exp\big(-\frac{1}{2\sigma^2}(X-\xi)^2\big),
\end{equation*}
where $0\leq \omega \leq 1$, $\mu\in \mathbb{R}$ and $\sigma^2\in \mathbb{R}^+$.
First, we assume $\omega=1/2$ and consider testing the hypotheses
\begin{equation}
    H: \xi=0,\sigma=1\quad \text{vs.} \quad K: \xi\neq 0 \text{ or } \sigma \neq 1.
    \label{mixturehy1}
\end{equation}
For this testing problem, the likelihhood function is unbounded under the alternative hypothesis.
In fact, if we take $\xi=X_1$ and let $\sigma^2\to 0$, then the likelihood tends to infinity.
Thus, the LRT can not be defined.
Using Theorem~\ref{Thm:maintheorem} and Proposition~\ref{Theoremless1}, we can obtain the following proposition.
\begin{proposition}
For hypotheses testing problem~\eqref{mixturehy1}, 
the generalized FBF satisfies
\begin{equation*}
    2\log \Lambda_{a,b} 
\end{equation*}
where $a>0$, $b>0$ and $a+b<1$.
    \label{propositionTT}
\end{proposition}
This example shows that the ILRT methodology has a wider application scope than the LRT.
\begin{proof}[\textbf{Proof of Proposition~\ref{propositionTT}}]
We shall verify Assumption~\ref{Assumption1} and Assumption~\ref{Assumption4}.
We use the following parameterization
$
\theta=(\xi,\tau)^T =(\xi,\sigma^{-2})^T$.
Then
\begin{equation*}
    p(X|\theta)=\frac{1}{2}\phi(X)+\frac{1}{2}\sqrt{\tau} \phi(\sqrt{\tau}(X-\xi)).
\end{equation*}
By direct calculation, we have 
\begin{equation*}
    \dot{\ell}_{\theta_0}(X)=\left(\frac{1}{2}X,\frac{1}{4}(1-X^2)\right)^T.
\end{equation*}
Hence $P^n_{\theta_0}\dot{\ell}_{\theta_0}=0_2$ and
\begin{equation*}
    I_{\theta_0}=\begin{pmatrix}
        \frac{1}{4} & 0\\
        0& \frac{1}{8}
    \end{pmatrix}.
\end{equation*}
Let $M>0$ is a fixed constant.  For $h=(h_1,h_2)^T\in \mathbb{R}^2$ such that $\|h\|\leq M$ and $i=1,\ldots, n$, we have
\begin{equation*}
    \frac{p(X_i|\theta_0+n^{-1/2}h)}{p(X_i|\theta_0)}=
    \frac{1}{2}+\frac{1}{2}\sqrt{1+\frac{h_2}{\sqrt{n}}} \exp\left\{-\frac{h_2}{2\sqrt{n}}X_i^2+\left(1+\frac{h_2}{\sqrt{n}}\right) \frac{h_1}{\sqrt{n}} X_i-\frac{1}{2}\left(1+\frac{h_2}{\sqrt{n}}\right)\frac{h_1^2}{n}\right\}.
\end{equation*}
It is known that $\max_{1\leq i\leq n}|X_i|=O_P(\sqrt{\log n})$. Thus, uniforly for$\|h\|\leq M$ and $i=1,\ldots,nj$, we have
\begin{align*}
    &\exp\left\{-\frac{h_2}{2\sqrt{n}}X_i^2+\left(1+\frac{h_2}{\sqrt{n}}\right) \frac{h_1}{\sqrt{n}} X_i-\frac{1}{2}\left(1+\frac{h_2}{\sqrt{n}}\right)\frac{h_1^2}{n}\right\}
    \\
    =&
    1-\frac{h_2}{2\sqrt{n}}X_i^2+\left(1+\frac{h_2}{\sqrt{n}}\right) \frac{h_1}{\sqrt{n}} X_i-\frac{1}{2}\left(1+\frac{h_2}{\sqrt{n}}\right)\frac{h_1^2}{n}
    \\
    &+\frac{1}{2}\left\{
    -\frac{h_2}{2\sqrt{n}}X_i^2+\left(1+\frac{h_2}{\sqrt{n}}\right) \frac{h_1}{\sqrt{n}} X_i-\frac{1}{2}\left(1+\frac{h_2}{\sqrt{n}}\right)\frac{h_1^2}{n}
\right\}^2+O_P\left(\frac{\log^3 n}{n^{3/2}}\right)
    \\
    %=&
    %1-\frac{h_2}{2\sqrt{n}}X_i^2+\left(1+\frac{h_2}{\sqrt{n}}\right) \frac{h_1}{\sqrt{n}} X_i-\frac{h_1^2}{2n}
    %+\frac{h_2^2}{8n}X_i^4 
    %+
     %\frac{h_1^2}{2n} X_i^2
     %-\frac{h_1 h_2}{2n}X_i^3
%+O_P\left(\frac{\log^3 n}{n^{3/2}}\right)
%\\
    =&
1
-\frac{h_1^2}{2n}
+\left(\frac{h_1}{\sqrt{n}}+\frac{h_1 h_2}{n}\right)  X_i
+
    \left(
        -\frac{h_2}{2\sqrt{n}}
    +
     \frac{h_1^2}{2n}
\right)
    X_i^2
     -\frac{h_1 h_2}{2n}X_i^3
    +\frac{h_2^2}{8n}X_i^4 
    +O_P\left(\frac{\log^3 n}{n^{3/2}}\right).
\end{align*}
On the other hand, 
\begin{equation*}
    \sqrt{1+\frac{h_2}{\sqrt{n}}}=1+\frac{h_2}{2\sqrt{n}}-\frac{h_2^2}{8n} +O\left(\frac{1}{n^3}\right).
\end{equation*}
Hence
\begin{equation*}
    \begin{split}
    &\sqrt{1+\frac{h_2}{\sqrt{n}}} \exp\left\{-\frac{h_2}{2\sqrt{n}}X_i^2+\left(1+\frac{h_2}{\sqrt{n}}\right) \frac{h_1}{\sqrt{n}} X_i-\frac{1}{2}\left(1+\frac{h_2}{\sqrt{n}}\right)\frac{h_1^2}{n}\right\}
    \\
    =&
    1
+\frac{h_1}{\sqrt{n}} X_i
+\left(\frac{h_2}{2\sqrt{n}}-\frac{h_1^2}{2n}\right)
        (1-X_i^2)
    +\frac{h_2^2}{8n}X_i^4 
    -\frac{h_2^2}{4n}X_i^2
-\frac{h_2^2}{8n}
+\frac{3 h_1 h_2}{2 n}  X_i
     -\frac{h_1 h_2}{2n}X_i^3
    +O_P\left(\frac{\log^3 n}{n^{3/2}}\right).
    \end{split}
\end{equation*}
Then
\begin{equation*}
    \begin{split}
    &\frac{p(X_i|\theta_0+n^{-1/2}h)}{p(X_i|\theta_0)}
    \\
    =&
1
+\frac{h_1}{2\sqrt{n}} X_i
+\left(\frac{h_2}{4\sqrt{n}}-\frac{h_1^2}{4n}\right)
        (1-X_i^2)
    +\frac{h_2^2}{16n}X_i^4 
    -\frac{h_2^2}{8n}X_i^2
-\frac{h_2^2}{16n}
+\frac{3 h_1 h_2}{4 n}  X_i
     -\frac{h_1 h_2}{4 n}X_i^3
    +O_P\left(\frac{\log^4 n}{n^{3/2}}\right).
    \end{split}
\end{equation*}
Using expansion $\log(1+x)=x-x^2/2+O(x^3)$ for $x\in(-1,1)$, we have
\begin{equation*}
    \begin{split}
    &\log \frac{p(X_i|\theta_0+n^{-1/2}h)}{p(X_i|\theta_0)}
    \\
    =&
\frac{h_1}{2\sqrt{n}} X_i
+\left(\frac{h_2}{4\sqrt{n}}-\frac{h_1^2}{4n}\right)
        (1-X_i^2)
    +\frac{h_2^2}{16n}X_i^4 
    -\frac{h_2^2}{8n}X_i^2
-\frac{h_2^2}{16n}
+\frac{3 h_1 h_2}{4 n}  X_i
     -\frac{h_1 h_2}{4 n}X_i^3
     \\
     &-\frac{h_1^2}{8n}X_i^2
     -\frac{h_2^2}{32n}(1-X_i^2)^2
     -\frac{h_1 h_2}{8n}X_i(1-X_i^2)
    +O_P\left(\frac{\log^6 n}{n^{3/2}}\right).
    \end{split}
\end{equation*}

\end{proof}


Next, we assume $\sigma^2=1$ is known and consider testing the hypotheses
\begin{equation}
    \omega \xi=0
    \quad \text{vs.}\quad
    \omega \xi \neq 0.
    \label{newHy}
\end{equation}
Even for this simple case, the likelihood ratio test also has undesirable behavior. In particular, it has trivial power under $n^{-1/2}$ local alternative hypothesis. See ,for example,~\cite{HALL2005158}.


We use FBF with $a+b<1$. The prior of $\omega$ is $\text{Beta}(\alpha_1,\alpha_2)$.
We have the following theorem.

\begin{theorem}
    Suppose $\pi(\omega,\xi)=\pi_{\omega}(\omega) \pi_{\xi}(\xi)$, $\pi_\xi(\xi)$ is positive and continuous at $\xi=0$,
    $\pi_\omega(\omega)\sim \text{Beta}(\alpha_1,\alpha_2)$ with $\alpha_1>1$.
    Suppose $a+b<1$.
    Then,
    \begin{enumerate}[(i)]
        \item
    under the null hypothesis,
    \begin{equation*}
        2\log \Lambda_{a,b} (\BX^{(n)})\overset{P^n_{\theta_0}}{\rightsquigarrow}\log(1+\frac a b)+ a\chi^2(1);
    \end{equation*}
\item
    suppose for some $s<1/4$, $\omega \geq n^{-s}$ for large $n$, $\sqrt{n}\omega \xi \to \eta$, then
    \begin{equation*}
        2\log \Lambda_{a,b} (\BX^{(n)})\overset{P^n_{\theta_n}}{\rightsquigarrow}\log(1+\frac a b)+ a\chi^2(1,\eta).
    \end{equation*}
\end{enumerate}
    \label{mixtureThm}
\end{theorem}
Theorem~\ref{mixtureThm} shows that under certain $\sqrt{n}$ local alternatives, ILRT has nontrival power. 
This illustrates the superiority of ILRT over likelihood ratio test.






\section{Conclusion}
In this paper, we proposed a flexible methodology ILRT which includes some existing method as special cases.
We gave the asymptotic distribution of the generalized FPF, which is a special case of ILRT.
We also investigates the asymptotic behavior of ILRT for general weight functions.
This allows one to use a simple form approximation of the posterior distribution as weight function.
In particular, we show that the weight function can be obtained from R\'{e}nyi divergence variational inference.

\section*{Acknowledgements}
This work was supported by the National Natural Science Foundation of China under Grant No. 11471035, 11471030.















\begin{appendices}

%\begin{lemma}[\cite{Kleijn2012The}, Lemma 2.1.]\label{Thm:localExpansion}
    %Under Assumption~\ref{Assumption1},
    %we have $\|\dot{\ell}_{\theta_0}(X)\|\leq m(X)$ $P_{\theta_0}$-a.s., $P_{\theta_0} \dot{\ell}_{\theta_0}(X)=0$ and for every $M>0$
    %\begin{equation*}
        %\sup_{\|h\|\leq M}\Big|
         %\log \frac{p_n(\BX^{(n)}|\theta_0+n^{-1/2}h)}{p_n(\BX^{(n)}|\theta_0)}-h^T I_{\theta_0}\Delta_{n,\theta_0}+\frac{1}{2}h^T I_{\theta_0}h
         %\Big|\xrightarrow{P^n_{\theta_0}}0.
    %\end{equation*}
    %%(See~\cite{van2000asymptotic} Theorem 5.23 or)
%\end{lemma}

%\begin{lemma}\label{Thm:someTest}
    %Under Assumptions~\ref{Assumption1} and~\ref{Assumption2},
    %there exists for every $M_n\to \infty$ a sequence of tests $\phi_n$ and a constant $\delta>0$ such that, for every sufficiently large $n$ and every $\|\theta-\theta_0\|\geq M_n/\sqrt{n}$,
    %$$
    %P^n_{\theta_0} \phi_n\to 0,\quad
    %P^n_{\theta} (1-\phi_n)\leq \exp[-\delta n(\|\theta-\theta\|^2\wedge 1)].
    %$$
    %(See~\cite{van2000asymptotic} Lemma 10.3.,~\cite{Kleijn2012The})
%\end{lemma}
    \section{Proofs in Section 2}

    \begin{proof}[\textbf{Proof of Theorem~\ref{Thm:maintheorem}}]
        For fixed $t>0$ and $M>0$, we have
\begin{align*}
    &\log \int_{\{\theta:\|\theta-\theta_0\|\leq M/\sqrt{n}\}}\big[ p_n(\BX^{(n)}|\theta)\big]^t \pi(\theta)\, d\theta\\
    =
    &\log \int_{\{\theta:\|\theta-\theta_0\|\leq M/\sqrt{n}\}}\big[ p_n(\BX^{(n)}|\theta)\big]^t \, d\theta+\log \pi(\theta_0)+o_{P^n_{\theta_0}}(1)\\
    =
    &\log \int_{\{h:\|h\|\leq M\}}\exp\big[ t\log p_n(\BX^{(n)}|\theta_0+n^{-1/2}h)\big] \, dh-\frac{p}{2}\log n+\log \pi(\theta_0)+o_{P^n_{\theta_0}}(1),
\end{align*}
where the first equality holds since $\pi(\theta)$ is continuous at $\theta_0$ and the second equality follows from the coordinate transformation $h=\sqrt{n}(\theta-\theta_0)$.
By the uniform expansion given by Assumption~\ref{Assumption1} and a little algebra, we have
\begin{align*}
    &\log \int_{\{h:\|h\|\leq M\}}\exp\big[ t\log p_n(\BX^{(n)}|\theta_0+n^{-1/2}h)\big] \, dh\\
    %=&\log \int_{\{h:\|h\|\leq M\}}\exp\big[ t\log p_n(\BX^{(n)}|\theta_0)+t h^T I_{\theta_0}\Delta_{n,\theta_0}-\frac{t}{2}h^T I_{\theta_0}h\big] \, dh+o_{P_{\theta_0}^n}(1)\\
    =&\log \int_{\{h:\|h\|\leq M\}}\exp\big[ -\frac{t}{2}(h-\Delta_{n,\theta_0})^T I_{\theta_0}(h-\Delta_{n,\theta_0})\big] \, dh
    +
    \frac{t}{2}\Delta_{n,\theta_0}^T I_{\theta_0}\Delta_{n,\theta_0}
    +
    t\log p_n(\BX^{(n)}|\theta_0)
    +o_{P_{\theta_0}^n}(1).
\end{align*}
        Thus
\begin{align*}
    &\log \int_{\{\theta:\|\theta-\theta_0\|\leq M/\sqrt{n}\}}\big[ p_n(\BX^{(n)}|\theta)\big]^t \pi(\theta)\, d\theta\\
    =
    &\log \int_{\{h:\|h\|\leq M\}}\exp\big[ -\frac{t}{2}(h-\Delta_{n,\theta_0})^T I_{\theta_0}(h-\Delta_{n,\theta_0})\big] \, dh
    \\
    & +
    \frac{t}{2}\Delta_{n,\theta_0}^T I_{\theta_0}\Delta_{n,\theta_0}
    +
    t\log p_n(\BX^{(n)}|\theta_0)
    -\frac{p}{2}\log n+\log \pi(\theta_0)+o_{P^n_{\theta_0}}(1).
\end{align*}
This equality holds for every $M>0$ and hence also for some $M_n\to \infty$.
        %By central limit theorem, $\Delta_{n,\theta_0}$ weakly converges to $N_p(\mathbf{0}_p,I_{\theta_0}^{-1})$ in $P_{\theta_0}^n$.
Since $\Delta_{n,\theta_0}$ is bounded in probability, we have
        $$
            \begin{aligned}
            &\log \int_{\{h:\|h\|\leq M_n\}}\exp\big[ -\frac{t}{2}(h-\Delta_{n,\theta_0})^T I_{\theta_0}(h-\Delta_{n,\theta_0})\big] \, dh
                \\
                =&
                \log \int_{\mathbb{R}^p}\exp\big[ -\frac{t}{2}(h-\Delta_{n,\theta_0})^T I_{\theta_0}(h-\Delta_{n,\theta_0})\big] \, dh+o_{P^n_{\theta_0}}(1)
                \\
                =&
                \frac{p}{2}\log(2\pi)-\frac{p}{2}\log t-\frac{1}{2}\log |I_{\theta_0}|
+o_{P^n_{\theta_0}}(1).
            \end{aligned}
        $$
        Thus,
$$
\begin{aligned}
    &\log \int_{\{\theta:\|\theta-\theta_0\|\leq M_n/\sqrt{n}\}}\big[ p_n(\BX^{(n)}|\theta)\big]^t \pi(\theta)\, d\theta\\
    =
    &
        \frac{p}{2}\log\big(\frac{2\pi}{n}\big)-\frac{p}{2}\log t-\frac{1}{2}\log |I_{\theta_0}|
        +\log \pi(\theta_0)
     +
    \frac{t}{2}\Delta_{n,\theta_0}^T I_{\theta_0}\Delta_{n,\theta_0}
    +
    t\log p_n(\BX^{(n)}|\theta_0)
    +o_{P^n_{\theta_0}}(1).
\end{aligned}
$$
If $L_t$ is consistent, then
\begin{align*}
    &\log L_t=\log \int_{\Theta}\big[ p_n(\BX^{(n)}|\theta)\big]^t \pi(\theta)\, d\theta\\
    =
    &
        \frac{p}{2}\log\big(\frac{2\pi}{n}\big)-\frac{p}{2}\log t-\frac{1}{2}\log |I_{\theta_0}|
        +\log \pi(\theta_0)
     +
    \frac{t}{2}\Delta_{n,\theta_0}^T I_{\theta_0}\Delta_{n,\theta_0}
    +
    t\log p_n(\BX^{(n)}|\theta_0)
    +o_{P^n_{\theta_0}}(1).
\end{align*}
Similarly, if $L_t^{(0)}$ is consistent,
\begin{equation*}
\begin{split}
    &\log L_t^{(0)} =\log \int_{\tilde{\Theta}_0}\big[ p_n(\BX^{(n)}|\nu,\xi_0)\big]^t \pi(\nu)\, d\nu\\
    =&
    \frac{p_0}{2}\log\big(\frac{2\pi}{n}\big)-\frac{p_0}{2}\log t-\frac{1}{2}\log |I_{\theta_0}^{(0)}|
                +\log \pi(\nu_0)
             +
             \frac{t}{2}\Delta_{n,\theta_0}^{{(0)}T} I^{(0)}_{\theta_0}\Delta^{(0)}_{n,\theta_0}
            +
            t\log p_n(\BX^{(n)}|\theta_0)
            +o_{P^n_{\theta_0}}(1).
\end{split}
\end{equation*}
By the mutual contiguity of $P_{\theta_0}^n$ and $P^n_{\theta_n}$, the term $o_{P^n_{\theta_0}}(1)$ is also $o_{P^n_{\theta_n}}(1)$. Hence
$$
\begin{aligned}
\log \Lambda_{a,b}
    =&
    \log L_{a+b}-
    \log L_b
    -
    \log L^{(0)}_{a+b}+
    \log L^{(0)}_b\\
    =&
    -\frac{p-p_0}{2}\log (1+\frac{a}{b})
    +
    \frac{a}{2}\Big(
    \Delta_{n,\theta_0}^T I_{\theta_0} \Delta_{n,\theta_0}
    -
    \Delta_{n,\theta_0}^{{(0)}T} I^{(0)}_{\theta_0} \Delta^{(0)}_{n,\theta_0}
    \Big)
    +o_{P^n_{\theta_n}}(1).
\end{aligned}
$$
Since $I_{\theta_0}^{(0)}= J^T I_{\theta_0}J$ and $\Delta_{n,\theta_0}^{(0)}=(J^T I_{\theta_0}J)^{-1} J^T I_{\theta_0} \Delta_{n,\theta_0}$, we have
$$
            \Delta_{n,\theta_0}^T I_{\theta_0} \Delta_{n,\theta_0}
            -
            \Delta_{n,\theta_0}^{{(0)}T} I^{(0)}_{\theta_0} \Delta^{(0)}_{n,\theta_0}
            =
            \Delta_{n,\theta_0}^T I_{\theta_0}^{1/2}\big(
            I_p-
            I_{\theta_0}^{1/2} J (J^T I_{\theta_0} J)^{-1} J^T I_{\theta_0}^{1/2}
            \big)I_{\theta_0}^{1/2}\Delta_{n,\theta_0},
$$
where $
            I_p-
            I_{\theta_0}^{1/2} J (J^T I_{\theta_0} J)^{-1} J^T I_{\theta_0}^{1/2}
$
is a projection matrix with rank $p-p_0$.
It remains to derive the asymptotic distribution of $\Delta_{n,\theta_0}$.
Let $h_n=\sqrt{n}(\theta_n-\theta_0)$.
     By Assumption~\ref{Assumption1} and CLT,
\begin{equation*}
    \begin{split}
    \left(
    \begin{matrix}
        \displaystyle
        \frac{1}{\sqrt{n}}\sum^n_{i=1}\dot{\ell}_{\theta_0}(X_i)
        \\
        \displaystyle
        \log \frac{p_n(\BX^{(n)}|\theta_n)}{p_n(\BX^{(n)}|\theta_0)}
    \end{matrix}
    \right)
    %&=\left(
        %\begin{matrix}
        %\frac{1}{\sqrt{n}}\sum^n_{i=1}\dot{\ell}_{\theta_0}(X_i)
        %\\
        %\frac{1}{\sqrt{n}}\sum^n_{i=1}h_n^T\dot{\ell}_{\theta_0}(X_i)-\frac{1}{2}h_n^T I_{\theta_0}h_n
        %\end{matrix}
    %\right)
    %+o_{P_0^n}(1)\\
    &\overset{P_0^n}{\rightsquigarrow}
    \mathcal{N}\left(
    \left(
    \begin{matrix}
        0\\
        -\frac{1}{2}\eta^T I_{\theta_0}\eta
    \end{matrix}
    \right),
    \left(
        \begin{matrix}
            I_{\theta_0}&I_{\theta_0}\eta\\
            \eta^T I_{\theta_0}&\eta^T I_{\theta_0}\eta
        \end{matrix}
    \right)
    \right).
    \end{split}
\end{equation*}
Hence by Le Cam's third lemma,
\begin{equation*}
    \frac{1}{\sqrt{n}}\sum^n_{i=1}\dot{\ell}_{\theta_0}(X_i)\overset{P^n_{\theta_n}}{\rightsquigarrow} \mathcal{N}(I_{\theta_0}\eta,I_{\theta_0}).
\end{equation*}
Consequently,
$
\Delta_{n,\theta_0}
$
weakly converges to $\mathcal{N}(\eta, I_{\theta_0}^{-1})$ in  $P^n_{\theta_n}$.
It follows that
\begin{equation*}
\Delta_{n,\theta_0}^T I_{\theta_0} \Delta_{n,\theta_0}
-
\Delta_{n,\theta_0}^{{(0)}T} I^{(0)}_{\theta_0} \Delta^{(0)}_{n,\theta_0}
\overset{P_{\eta_n}^n}{\rightsquigarrow} \chi^2(p-p_0,\delta),
\end{equation*}
which completes the proof.

    \end{proof}

\begin{proof}[\textbf{Proof of Proposition~\ref{exponentialCon}}]
    For exponential family, we have
    $$
    I_{\theta_0}\Delta_{n,\theta_0}=n^{-1/2}\sum_{i=1}^n T(X_i)-\sqrt{n}\frac{\partial}{\partial \theta} A(\theta_0)
    $$
    and
    $$
    \log\frac{p_n(\BX^{(n)}|\theta_0+n^{-1/2}h)}{p_n(\BX^{(n)}|\theta_0)}
    =h^T I_{\theta_0} \Delta_{n,\theta_0}-\frac{1}{2} h^T I_{\theta_0} h-
    g_n(h),
    $$
    where
    $$
    g_n(h)=n\Big(A(\theta_0+n^{-1/2}h)-A(\theta_0)-n^{-1/2}h \frac{\partial}{\partial \theta}A(\theta_0)-\frac{1}{2n}h^T I_{\theta_0}h\Big).
    $$
    Without loss of generality, we assume $M_n\to \infty$ and $M_n^3/\sqrt{n}\to 0$.
    By Taylor's theorem and the continuity of the third derivative of $A(\theta)$, 
    $$
        \max_{\{h:\|h\|\leq M_n\}}|g_n(h)|=O\left(\frac{M_n^3}{\sqrt{n}}\right)\to 0.
    $$
    This allows us to derive the following lower bound for $L_t$.
\begin{align*}
    L_t
    \geq &
    \int_{\{\theta:\|\theta-\theta_0\|\leq M_n/\sqrt{n}\}} \big[p_n(\BX^{(n)}|\theta)\big]^t \pi(\theta)\, d\theta
    \\
    =&
    (1+o_{P_{\theta_0}^n}(1))n^{-p/2}\pi(\theta_0)\big[p_n(\BX^{(n)}|\theta_0)\big]^t\int_{\{h:h\leq M_n\}} \exp\big[ t h^T I_{\theta_0}\Delta_{n,\theta_0}-\frac{t}{2}h^T I_{\theta_0}h\big] \, dh
    \\
    =&
    (1+o_{P_{\theta_0}^n}(1))n^{-p/2}\pi(\theta_0) \big[p_n(\BX^{(n)}|\theta_0)\big]^t\int_{\mathbb{R}^p} \exp\big[t h^T I_{\theta_0}\Delta_{n,\theta_0}-\frac{t}{2}h^T I_{\theta_0}h\big] \, dh
    \\
    =&
    (1+o_{P_{\theta_0}^n}(1))n^{-p/2}\pi(\theta_0)\big[p_n(\BX^{(n)}|\theta_0)\big]^t
    \exp\big[-\frac{t}{2}\Delta_{n,\theta_0}^T I_{\theta_0}\Delta_{n,\theta_0}\big]
    (2\pi)^{p/2} t^{-p/2}  |I_{\theta_0}|^{-1/2}.
\end{align*}

Next we upper bound $\log (p_n(\BX^{(n)}|\theta)/p_n(\BX^{(n)}|\theta_0))$ for $\|\theta-\theta_0\|\geq M_n/\sqrt{n}$.
    We have
    $$
    \begin{aligned}
        &\max_{\{\theta:\|\theta-\theta_0\|=M_n/\sqrt{n}\}}
    \log\frac{p_n(\BX^{(n)}|\theta)}{p_n(\BX^{(n)}|\theta_0)}
    =
    \max_{\{h:\|h\|=M_n\}}
    \log\frac{p_n(\BX^{(n)}|\theta_0+n^{-1/2}h)}{p_n(\BX^{(n)}|\theta_0)}
        \\
        &\leq
         \|I_{\theta_0}\Delta_{n,\theta_0}\| M_n -\frac{\lambda_{\min}(I_{\theta_0})}{2} M_n^2+
        \max_{\{h:\|h\|=M_n\}}|g_n(h)|,
    \end{aligned}
    $$
    where $\lambda_{\min}(I_{\theta_0})>0$ is the minimum eigenvalue of $I_{\theta_0}$.
    Also note that $I_{\theta_0}\Delta_{n,\theta_0}$ is bounded in probability. Hence with probability tending to $1$,
    $$
    \begin{aligned}
        &\max_{\{\theta:\|\theta-\theta_0\|=M_n/\sqrt{n}\}}
    \log\frac{p_n(\BX^{(n)}|\theta)}{p_n(\BX^{(n)}|\theta_0)}
        \leq 
        -\frac{\lambda_{\min}(I_{\theta_0})}{4}M_n^2.
    \end{aligned} 
    $$
    By the concavity of $\log p_n(\BX^{(n)}|\theta)$, for $\|\theta-\theta_0\|\geq M_n/\sqrt{n}$,
    $$
     \frac{M_n/\sqrt{n}}{\|\theta-\theta_0\|}
     \Big(
     \log p_n(\BX^{(n)}|\theta)-\log p_n(\BX^{(n)}|\theta_0)
     \Big)
     \leq
     \log p_n \Big(\BX^{(n)}\Big|\theta_0+\frac{M_n/\sqrt{n}}{\|\theta-\theta_0\|}(\theta-\theta_0)\Big)-\log p_n(\BX^{(n)}|\theta_0).
    $$
    Thus,
    $$
    \begin{aligned}
     \log \frac{p_n(\BX^{(n)}|\theta)}{ p_n(\BX^{(n)}|\theta_0)}
        &\leq
        \frac{\sqrt{n}\|\theta-\theta_0\|}{M_n}
     \log \frac{p_n\Big(\BX^{(n)}\Big|\theta_0+\frac{M_n/\sqrt{n}}{\|\theta-\theta_0\|}(\theta-\theta_0)\Big)}{ p_n(\BX^{(n)}|\theta_0)}
        \\
        &\leq
        \frac{\sqrt{n}\|\theta-\theta_0\|}{M_n}
        \Big(-\frac{\lambda_{\min}(I_{\theta_0})}{4}M_n^2\Big)
        \\
        &=
        -\frac{\lambda_{\min}(I_{\theta_0})}{4}\sqrt{n}\|\theta-\theta_0\|
        M_n.
    \end{aligned}
    $$
    Fix an $\epsilon>0$ such that $\sup_{\|\theta-\theta_0\|< \epsilon}\pi(\theta) < +\infty $. We have
$$
    \begin{aligned}
        &\int_{\{\theta:\|\theta-\theta_0\|> M_n/\sqrt{n}\}} \big[p_n(\BX^{(n)}|\theta)\big]^t \pi(\theta)\, d\theta
        \\
        \leq&
        \big[p_n(\BX^{(n)}|\theta_0)\big]^t 
        \int_{\{\theta:\|\theta-\theta_0\|> M_n/\sqrt{n}\}} 
        \exp\Big[-\frac{t\lambda_{\min}(I_{\theta_0})}{4}\sqrt{n}\|\theta-\theta_0\|M_n\Big]
        \pi(\theta)\, d\theta
        \\
        =&
        \big[p_n(\BX^{(n)}|\theta_0)\big]^t 
        \Big(
        \int_{\{\theta:M_n/\sqrt{n}\leq \|\theta-\theta_0\|\leq \epsilon \}} 
        \exp\Big[-\frac{t\lambda_{\min}(I_{\theta_0})}{4}\sqrt{n}\|\theta-\theta_0\|M_n\Big]
        \pi(\theta)\, d\theta
        \\
        &+
        \int_{\{\theta:\|\theta-\theta_0\|> \epsilon\}} 
        \exp\Big[-\frac{t\lambda_{\min}(I_{\theta_0})}{4}\sqrt{n}\|\theta-\theta_0\|M_n\Big]
        \pi(\theta)\, d\theta
        \Big)
        \\
        \leq& 
        \big[p_n(\BX^{(n)}|\theta_0)\big]^t 
        \Big(
        \big(\sup_{\|\theta-\theta_0\|<\epsilon}\pi(\theta)\big)
        \int_{\{\theta: \|\theta-\theta_0\|\geq M_n/\sqrt{n}\}} 
        \exp\Big[-\frac{t\lambda_{\min}(I_{\theta_0})}{4}\sqrt{n}\|\theta-\theta_0\|M_n\Big]
        \, d\theta
        \\
        &+
        \exp\Big[-\frac{t\lambda_{\min}(I_{\theta_0})}{4}\epsilon\sqrt{n}M_n\Big]
        \Big)
        \\
        =& 
        \big[p_n(\BX^{(n)}|\theta_0)\big]^t 
        \Big(
        \big(\sup_{\|\theta-\theta_0\|<\epsilon}\pi(\theta)\big)
        n^{-p/2}
        \int_{\{h: \|h\|\geq M_n\}} 
        \exp\Big[-\frac{t\lambda_{\min}(I_{\theta_0})}{4}\|h\| M_n\Big]
        \, dh
        \\
        &+
        \exp\Big[-\frac{t\lambda_{\min}(I_{\theta_0})}{4}\epsilon\sqrt{n}M_n\Big]
        \Big).
    \end{aligned}
$$

Thus,
$$
    \begin{aligned}
        &\frac{
            \int_{\{\theta:\|\theta-\theta_0\|> M_n/\sqrt{n}\}} \big[p_n(\BX^{(n)}|\theta)\big]^t \pi(\theta)\, d\theta
        }
        {
            \int_{\Theta} \big[p_n(\BX^{(n)}|\theta)\big]^t \pi(\theta)\, d\theta
        }
        \\
        =&
        O_{P_{\theta_0}^n}(1)
        \Big(
        \int_{\{h: \|h\|\geq M_n\}} 
        \exp\Big[-\frac{t\lambda_{\min}(I_{\theta_0})}{4}\|h\| M_n\Big]
        \, dh
        +
        n^{p/2}\exp\Big[-\frac{t\lambda_{\min}(I_{\theta_0})}{4}\epsilon\sqrt{n}M_n\Big]
        \Big)
        \\
        =&o_{P^n_{\theta_0}}(1).
    \end{aligned}
$$

\end{proof}

\begin{proof}[\textbf{Proof of Proposition~\ref{exprop}}]

Note that $L_1$ is well defined $P_{\theta_0}^n$-a.s.\ since it has finite integral
$$
\int_{\mathcal{X}^n} L_1 \, d\mu^n=
\int_{\Theta}\Big(\int_{\mathcal{X}^n} p_n(\BX^{(n)}|\theta) \, d\mu^n \Big) \, \pi(\theta)\, d\theta=1.
$$
For $0<t<1$, by H\"older's inequality, we have $L_{t}\leq L_1^{1/t}$. This proves the first part of the proposition. 

To prove the second part of the proposition, consider the following example.
Suppose $X_1,\ldots,X_n$ are iid from the density
$$
    p(X|\theta)=C |X-\theta|^{-\gamma}\exp\big[-(X-\theta)^2\big]
,
$$
where $C$ is the normalizing constant and $\gamma\in(0,1)$ is a known hyperparameter. The parameter space $\Theta$ is equal to $\mathbb{R}$.
    %We would like to test the hypotheses $H_0:\theta=0$ vs. $H_1:\theta\neq 0$.
    %The likelihood function is
    %$$
    %p_n(\BX^{(n)}|\theta)=C^n \Big[\prod_{i=1}^n |X_i-\theta|\Big]^{-\gamma}
    %\exp \big[-\sum_{i=1}^n (X_i-\theta)^2 \big].
    %$$
%It can be seen that the likelihood tends to infinity as $\theta$ tends to any one of $X_1,\ldots, X_n$.
    %Consequently, LRT fails in this model.
    %We impose a prior $\pi(\theta)$.
    %Suppose that $\pi(\theta)$ is positive for all $\theta$.
Then
$$
    \begin{aligned}
        L_t=&
    C^n \int_{-\infty}^{+\infty}
\Big[\prod_{i=1}^n |X_i-\theta|\Big]^{-t\gamma}
    \exp \big[-t\sum_{i=1}^n (X_i-\theta)^2 \big]
        \pi(\theta)
    \,
    d \theta.
    \end{aligned}
$$
Note that almost surely, there is no tie among $X_1,\ldots,X_n$. Consequently, $L_t(\BX^{(n)})=+\infty$ almost surely if and only if $t\geq \gamma^{-1}$.
Since $\gamma^{-1}\in (1,+\infty)$, this example shows that $L_t$ is not always well defined for $t>1$.

\end{proof}



\begin{proof}[\textbf{Proof of Proposition~\ref{Theoremless1}}]
    Note that
       \begin{equation}\label{eq:numden}
       \frac{L_{t} (\{\theta: \|\theta-\theta_0\|\geq \frac{M_n}{\sqrt{n}}\})}
           {L_{t}}
=
    \frac{\displaystyle
        \int_{\{\theta: \|\theta-\theta_0\|\geq \frac{M_n}{\sqrt{n}}\}} \Big[ \frac{p_n(\BX^{(n)}|\theta)}{p_n(\BX^{(n)}|\theta_0)} \Big]^{t} \pi(\theta) \, d \theta
    }{\displaystyle
        \int_{\Theta} \Big[ \frac{p_n(\BX^{(n)}|\theta)}{p_n(\BX^{(n)}|\theta_0)} \Big]^{t} \pi(\theta) \, d \theta
    }.
       \end{equation}
    Without loss of generality, we assume ${M_n}/{\sqrt{n}}\to 0$.

    Consider the expactation of the numerator of~\ref{eq:numden}.
    It follows from Fubini's theorem that
\begin{align*}
    &P_{\theta_0}^n\int_{\{\theta:\|\theta-\theta_0\|\geq \frac{M_n}{\sqrt{n}}\} } \Big[ \frac{p_n(\BX^{(n)}|\theta)}{p_n(\BX^{(n)}|\theta_0)}  \Big]^{t} \pi(\theta) \, d \theta
    \\
    =&
    \int_{\{\theta:\|\theta-\theta_0\|\geq \frac{M_n}{\sqrt{n}}\} } \left\{\int_{\mathcal{X}^n}\big[ {p_n} (\BX^{(n)}|\theta)\big]^t \big[ p_n (\BX^{(n)}|\theta_0) \big]^{1-t} \, d\mu^n \right\} \pi(\theta) \, d \theta\\
    =&
    \int_{\{\theta:\|\theta-\theta_0\|\geq \frac{M_n}{\sqrt{n}}\} } \big[ \rho_{t}(\theta,\theta_0) \big]^n \pi(\theta) \, d \theta\\
    = &
    \int_{\{\theta:\|\theta-\theta_0\|\geq \frac{M_n}{\sqrt{n}}\} } \exp \big[-(1-t) n D_t(\theta||\theta_0) \big] \pi(\theta) \, d \theta.
\end{align*}
    Decompose the integral region into two parts $\{\theta:\frac{M_n}{\sqrt{n}}\leq \|\theta-\theta_0\|\leq \delta \}$ and $\{\theta: \|\theta-\theta_0\|>\delta\}$.
Then Assumption~\ref{Assumption4} implies that
\begin{align*}
    &\int_{\{\theta:\|\theta-\theta_0\|\geq \frac{M_n}{\sqrt{n}}\} } 
    \exp \big[ -(1-t) {n} D_t(\theta||\theta_0) \big] \pi(\theta) \, d \theta
    \\
    =&\left(
        \int_{\{\theta:\frac{M_n}{\sqrt{n}}\leq \|\theta-\theta_0\|\leq \delta \}}
        +
\int_{\{\theta: \|\theta-\theta_0\|>\delta\}}
    \right)
    \exp\big[ -(1-t) {n} D_t(\theta||\theta_0) \big] \pi(\theta) \, d \theta
    \\
    \leq &
    \max_{\|\theta-\theta_0\|\leq \delta}\pi(\theta)
    \int_{\big\{\theta: \|\theta-\theta_0\|\geq \frac{M_n}{\sqrt{n}} \big\}}
    \exp\big[ -(1-t)C {n} \|\theta-\theta_0\|^2 \big]
    \, d \theta
    +
    \exp\big[ -(1-t)\epsilon n\big]
    \\
    =&
    \big(\max_{\|\theta-\theta_0\|\leq \delta}\pi(\theta)\big)
    n^{-p/2}\int_{\big\{h: \|h\|\geq M_n \big\}} \exp\big[-(1-t)C \|h\|^2 \big] \, d h
    +
    \exp\big[ -(1-t)\epsilon n\big].
\end{align*}
Hence
\begin{equation}\label{Prop4:eq2}
    n^{p/2}\int_{\{\theta:\|\theta-\theta_0\|\geq \frac{M_n}{\sqrt{n}}\} } \Big[ \frac{p_n(\BX^{(n)}|\theta)}{p_n(\BX^{(n)}|\theta_0)}  \Big]^{t} \pi(\theta) \, d \theta
    =o_{P^n_{\theta_0}}(1).
\end{equation}

    Now we consider the denominator of~\eqref{eq:numden}.
    $$
    \begin{aligned}
        & \int_{\Theta}\Big[\frac{p_n(\BX^{(n)}|\theta)}{p_n(\BX^{(n)}|\theta_0)}\Big]^{t} \pi(\theta)\, d\theta
        \geq
        \int_{\{\theta:\|\theta-\theta_0\|\leq n^{-1/2}\}}\Big[\frac{p_n(\BX^{(n)}|\theta)}{p_n(\BX^{(n)}|\theta_0)}\Big]^{t} \pi(\theta)\, d\theta
        \\
        \geq &
        \Big(
        \min_{\|\theta-\theta_0\|\leq n^{-1/2}} 
\Big[\frac{p_n(\BX^{(n)}|\theta)}{p_n(\BX^{(n)}|\theta_0)}\Big]^{t} \pi(\theta)
        \Big)
        \int_{\{\theta:\|\theta-\theta_0\|\leq n^{-1/2}\}}1\, d\theta\\
        \geq&
        \Big(
        \exp
\Big[
        t\min_{\|\theta-\theta_0\|\leq n^{-1/2}} 
        \log\frac{p_n(\BX^{(n)}|\theta)}{p_n(\BX^{(n)}|\theta_0)}
        \Big]
        \Big)
        \Big(\min_{\|\theta-\theta_0\|\leq n^{-1/2}} 
        \pi(\theta)
        \Big)
        n^{-p/2}\frac{2\pi^{p/2}}{\Gamma(p/2)}.
    \end{aligned}
    $$
    By Assumption~\ref{Assumption1},
    $$
   \begin{aligned} 
        \min_{\|\theta-\theta_0\|\leq n^{-1/2}} 
        \log\frac{p_n(\BX^{(n)}|\theta)}{p_n(\BX^{(n)}|\theta_0)}
        \geq
        -\|I_{\theta_0}\Delta_{n,\theta_0}\|-\frac{1}{2}\|I_{\theta_0}\|+
        o_{P^n_{\theta_0}}(1),
   \end{aligned}
    $$
    where
    $I_{\theta_0}\Delta_{n,\theta_0}$
    is bounded in probability.
    Also note that 
    $$\min_{\|\theta-\theta_0\|\leq n^{-1/2}} \pi(\theta)\to \pi(\theta_0)>0.$$
    Then for every $\epsilon'>0$, there is a constant $c>0$ such that with probability at least $1-\epsilon'$,
    $$
         \int_{\Theta}\Big[\frac{p_n(\BX^{(n)}|\theta)}{p_n(\BX^{(n)}|\theta_0)}\Big]^{t} \pi(\theta)\, d\theta\geq c n^{-p/2}.
    $$
    This inequality, together with~\eqref{Prop4:eq2}, proves the consistency of $L_t$.
\end{proof}



\begin{proof}[\textbf{Proof of Theorem~\ref{theoremMain}}]


    By contiguity, we only need to prove the convergence in $P_{\theta_0}^n$.
    Let $M>0$ be any fixed number.
     %In the first part, we prove that for any fixed positive number $M$,
    Assumption~\ref{Assumption1} implies that
\begin{equation}\label{eq:8}
    \begin{split}
    &\int_{\|h\|\leq M} \left[\frac{p_n(\BX^{(n)}|\theta_0+n^{-1/2}h)}{p_n(\BX^{(n)}|\theta_0)}\right]^a \pi_n (h;\BX^{(n)}) \, dh\\
    =&
    \exp [o_{p^n_0}(1)]\int_{\|h\|\leq M} \exp\big[ a h^T I_{\theta_0}\Delta_{n,\theta_0}- \frac{a}{2}h^T I_{\theta_0}h\big]\pi_n (h;\BX^{(n)}) \, dh.
\end{split}
\end{equation}
    %So we only need to consider $\int_{\|h\|\leq M} \exp\big[h^T I_{\theta_0}\Delta_{n,\theta_0}-\frac{1}{2}h^T I_{\theta_0}h\big]\pi_n (h;\BX^{(n)}) \, dh$.
    %By central limit theorem, $\Delta_{n,\theta_0}$ weakly converges to a normal distribution.
Since $\Delta_{n,\theta_0}$ is bounded in probability, we have
\begin{equation*}
\begin{aligned}
    &\int_{\|h\|\leq M} \exp\big[a h^T I_{\theta_0}\Delta_{n,\theta_0}-\frac{a}{2}h^T I_{\theta_0}h\big] \big|\pi_n (h;\BX^{(n)})-\phi(h;\Delta_{n,\theta_0},b^{-1} I_{\theta_0}^{-1})\big|\, dh
\\
    \leq& \sup_{\|h\|\leq M}\exp\big[a h^T I_{\theta_0}\Delta_{n,\theta_0}-\frac{a}{2}h^T I_{\theta_0}h\big] 
    \int_{\|h\|\leq M}
    \big|\pi_n (h;\BX^{(n)})-\phi(h;\Delta_{n,\theta_0}, b^{-1} I_{\theta_0}^{-1})\big|\, dh
    \xrightarrow{P^n_{\theta_0}}0.
\end{aligned}
\end{equation*}
This, combined with~\eqref{eq:8}, yields 
\begin{equation}\label{eq:14}
    \begin{split}
    &\int_{\|h\|\leq M} \left[ \frac{p_n(\BX^{(n)}|\theta_0+n^{-1/2}h)}{p_n(\BX^{(n)}|\theta_0)}\right]^a \pi_n (h;\BX^{(n)}) \, dh
    \\
    =&\int_{\|h\|\leq M} \exp\big[ a h^T I_{\theta_0}\Delta_{n,\theta_0}-\frac{a}{2}h^T I_{\theta_0}h\big] \phi(h;\Delta_{n,\theta_0}, b^{-1} I_{\theta_0}^{-1})\, dh
    +o_{P^n_{\theta_0}}(1).
\end{split}
\end{equation}
This is true for every $M>0$ and hence also for some $M_n\to \infty$.

Now we prove that for any $M_n\to +\infty$, we have
\begin{equation}\label{eq:4}
    \int_{\|h\|> M_n}\left[\frac{p_n(\BX^{(n)}|\theta_0+n^{-1/2}h)}{p_n(\BX^{(n)}|\theta_0)}\right]^{a} \pi_n(h;\BX^{(n)})\, dh
    \xrightarrow{P_{\theta_0}^n}0.
\end{equation}
By Assumption~\ref{Assumption3}, for any $\epsilon>0$, with probability at least $1-\epsilon$,
\begin{equation}\label{eq:4l}
    \int_{\|h\|> M_n}\left[\frac{p_n(\BX^{(n)}|\theta_0+n^{-1/2}h)}{p_n(\BX^{(n)}|\theta_0)}\right]^{a} \pi_n(h;\BX^{(n)})\, dh
    \leq
    \int_{\|h\|> M_n}\left[\frac{p_n(\BX^{(n)}|\theta_0+n^{-1/2}h)}{p_n(\BX^{(n)}|\theta_0)}\right]^{a} T(h)\, dh.
\end{equation}
By  H\"older's inequality, 
\begin{equation*}
    P^n_{\theta_0}\left[\frac{p_n(\BX^{(n)}|\theta_0+n^{-1/2}h)}{p_n(\BX^{(n)}|\theta_0)}\right]^{a}
    =
    \int p_n(\BX^{(n)}|\theta_0+n^{-1/2}h)^a p_n(\BX^{(n)}|\theta_0)^{1-a} d\mu^n\leq 1.
\end{equation*}
Hence the expectation of the right hand side of~\eqref{eq:4l} satisfies
\begin{equation*}
   P^n_{\theta_0} \int_{\|h\|> M_n}\left[\frac{p_n(\BX^{(n)}|\theta_0+n^{-1/2}h)}{p_n(\BX^{(n)}|\theta_0)}\right]^{a} T(h)\, dh
   \leq
   \int_{\|h\|> M_n} T(h)\, dh\to 0.
\end{equation*}
This verifies~\eqref{eq:4}.

Combining~\eqref{eq:14} and~\eqref{eq:4} yields
\begin{equation*}
    \int_{h\in\mathbb{R}^p} \left[ \frac{p_n(\BX^{(n)}|\theta_0+n^{-1/2}h)}{p_n(\BX^{(n)}|\theta_0)}\right]^a \pi_n (h;\BX^{(n)}) \, dh
    =\left(1+\frac{a}{b}\right)^{-p/2} \exp\left(\frac{a}{2}\Delta_{n,\theta_0}^T I_{\theta_0}\Delta_{n,\theta_0}\right)+o_{P_{\theta_0}^n}(1).
\end{equation*}
Similarly, we have
\begin{equation*}
    \int_{h^{(0)}\in\mathbb{R}^{p_0}} \left[ \frac{p_n(\BX^{(n)}|\nu_0+n^{-1/2}h^{(0)},\xi_0)}{p_n(\BX^{(n)}|\theta_0)}\right]^a \pi_n (h^{(0)};\BX^{(n)}) \, dh^{(0)}
    =\left(1+\frac{a}{b}\right)^{-p_0/2} \exp\left(\frac{a}{2}\Delta_{n,\theta_0}^{(0)T} I^{(0)}_{\theta_0}\Delta^{(0)}_{n,\theta_0}\right)+o_{P_{\theta_0}^n}(1).
\end{equation*}
Hence
\begin{equation*}
    \begin{aligned} 
        2\log \Lambda_{a,b}&=
        -{(p-p_0)}\log (1+\frac{a}{b})+{a}\left(
\Delta_{n,\theta_0}^T I_{\theta_0}\Delta_{n,\theta_0}
        -\Delta_{n,\theta_0}^{(0)T} I^{(0)}_{\theta_0}\Delta^{(0)}_{n,\theta_0}\right)
        +o_{P^n_{\theta_0}}(1).
    \end{aligned}
\end{equation*}
Then the conclusion follows by the same argument as the last part of Theorem~\ref{Thm:maintheorem}.
\end{proof}

\section{Proofs in Section 3}

We would like to derive the asymptotics of
\begin{equation}
    \int_\Theta \Big[\prod_{i=1}^n \frac{p(X_i|\omega,\xi)}{p_0(X_i)}\Big]^t \pi(\omega,\xi)\, d\omega d\mu.
    \label{temp}
\end{equation}

Let $A(M_n)=\{(\omega, \xi): \omega \big( 2\Phi(|\xi|/2)-1\big)\leq M_n n^{-1/2}\}$, we have
\begin{proposition}
    If $M_n \geq  \frac{1}{4(t\wedge (1-t))}\log n$,
    \begin{equation*}
        \myE \int_{A(M_n)^c} \Big[\prod_{i=1}^n \frac{p(X_i|\omega,\xi)}{p_0(X_i)}\Big]^t \pi(\omega,\xi)\, d\omega d\mu =o(n^{-1/2}).
    \end{equation*}
\end{proposition}
\begin{proof}
\begin{equation*}
    \myE \int_{A(M_n)^c} \Big[\prod_{i=1}^n \frac{p(X_i|\omega,\xi)}{p_0(X_i)}\Big]^t \pi(\omega,\xi)\, d\omega d\xi
    =
    \int_{A(M_n)^c} \big( \int p(X_1|\omega,\xi)^t p_0(X_1)^{1-t}\, d\mu\big)^n \pi(\omega,\xi)\, d\omega d\xi.
\end{equation*}
Note that
\begin{align*}
    &\int p(X_i|\omega,\xi)^t p_0(X_i)^{1-t}\, d\mu
    \\
    \leq & \Big(\int \sqrt{p(X_i|\omega,\xi) p_0(X_i)}\, d\mu\Big)^{2(t\wedge (1-t))}
    \\
= & \Big(1-\frac{1}{2}\int \big(\sqrt{p(X_i|\omega,\xi) }-\sqrt{p_0(X_i)}\big)^2\, d\mu\Big)^{2(t\wedge (1-t))}
\\
\leq & \exp \Big( -(t\wedge (1-t))\int \big(\sqrt{p(X_i|\omega,\xi) }-\sqrt{p_0(X_i)}\big)^2\, d\mu \Big)
\\
\leq & \exp \Big( -\frac{1}{2}(t\wedge (1-t)) \big(\int \big| p(X_i|\omega,\xi)-p_0(X_i)\big|\, d\mu \big)^2 \Big)
\\
= & \exp \Big( -\frac{1}{2}(t\wedge (1-t)) \omega^2 \big(\int \big| \phi(X_i -\xi)-\phi (X_i)\big|\, d\mu \big)^2 \Big)
\\
= & \exp \Big( -2(t\wedge (1-t)) \omega^2 \big( 2\Phi(|\xi|/2)-1\big)^2 \Big)
\\
\end{align*}
Thus,
\begin{align*}
    &\myE \int_{A(M_n)^c} \Big[\prod_{i=1}^n \frac{p(X_i|\omega,\xi)}{p_0(X_i)}\Big]^t \pi_{\omega}(\omega)\pi_{\xi}(\xi)\, d\omega d\xi
\\
    \leq
    &
    \int_{A(M_n)^c} \exp \Big( -2(t\wedge (1-t)) n\omega^2 \big( 2\Phi(|\xi|/2)-1\big)^2 \Big)
\pi_{\omega}(\omega) \pi_\xi(\xi)\, d\omega d\xi
\end{align*}

If $M_n \geq \frac{1}{4(t\wedge (1-t))}\log n$,
\begin{align*}
    &\myE \int_{A(M_n)^c} \Big[\prod_{i=1}^n \frac{p(X_i|\omega,\xi)}{p_0(X_i)}\Big]^t \pi_{\omega}(\omega)\pi_{\xi}(\xi)\, d\omega d\xi
\\
    \leq
    &
    n^{-1/2}
    \int_{A(M_n)^c} 
\pi_{\omega}(\omega) \pi_\xi(\xi)\, d\omega d\xi
=o(n^{-1/2}).
\end{align*}




\end{proof}


\begin{proof}[\textbf{Proof of Theorem~\ref{mixtureThm}}]
We have
\begin{equation*}
    \sum_{i=1}^n \big(\log p(X_i|\omega,\xi)-\log p_0(X_i)\big)
    =\sum_{i=1}^n \log\Big(1+\omega \big(\exp(\xi X_i -\xi^2/2)-1\big)\Big)=\sum_{i=1}^n \log(1+\omega \xi Y_i),
\end{equation*}
where
$
Y_i=\big(\exp(\xi X_i -\xi^2/2)-1\big)/\xi
$ if $\xi \neq 0$ and $Y_i=X_i$ if $\xi =0$.

Let $r>1$ and $s< 1/4$, on $A((\log n)^r)\cap \{\omega\geq n^{-s}\}$,
we have $|\xi| = O\big((\log n)^{r}/n^{1/2-s}\big)$.

It is known that $\max_{1\leq i \leq n}|X_i|=O_P(\sqrt{\log n})$.
On $A((\log n)^r)\cap \{\omega\geq n^{-s}\}$, we have $\max_{1\leq i\leq n}|\xi X_i-\xi^2/2|\leq |\xi| \max_{1\leq i\leq n}|X_i|+\xi^2/2=O_P(|\xi|(\log n)^{1/2})$.
Then on $A((\log n)^r)\cap \{\omega\geq n^{-s}\}$, uniformly for $i=1,\ldots, n$, we have
\begin{align*}
    Y_i&=\xi^{-1}\Big(\xi X_i-\xi^2/2 +\frac{1}{2}(\xi X_i-\xi^2/2)^2+O_P \big(|\xi|^3 (\log n)^{3/2}\big)\Big) 
    \\
    &=X_i-\frac{1}{2}\xi+\frac{1}{2} \xi X_i^2-\frac{1}{2} \xi^2 X_i +\frac{1}{8}\xi^3+O_P \big(|\xi|^2 (\log n)^{3/2}\big)
    \\
    &=X_i+\frac{1}{2} \xi (X_i^2-1) + O_P \big(|\xi|^2 (\log n)^{3/2}\big).
\end{align*}
In particular, on $A((\log n)^r)\cap \{\omega\geq n^{-s}\}$, we have $\max_{1\leq i \leq n}|Y_i|=O_P(\sqrt{\log n})$.
On $A((\log n)^r)\cap \{\omega\geq n^{-s}\}$, we have $\omega \xi =O((\log n)^r /\sqrt{n})$, then
\begin{align*}
    &\sum_{i=1}^n \log(1+\omega \xi Y_i)
    \\
    =& \omega \xi\sum_{i=1}^n Y_i -\frac{1}{2} \omega^2 \xi^2 \sum_{i=1}^n Y_i^2+O_P(n\omega^3 \xi^3 (\log n)^{3/2})
    \\
    =& \omega \xi\sum_{i=1}^n Y_i -\frac{1}{2} \omega^2 \xi^2 \sum_{i=1}^n Y_i^2+o_P(1).
\end{align*}
Note that
\begin{align*}
    &\omega \xi\sum_{i=1}^n Y_i
    =
    \omega \xi\sum_{i=1}^n 
    X_i+\frac{1}{2} \omega \xi^2\sum_{i=1}^n (X_i^2-1) + O_P \big(n\omega |\xi|^3 (\log n)^{3/2}\big)
    \\
    =&
    \omega \xi\sum_{i=1}^n X_i + O_P\Big( \frac{(\log n)^{3r+3/2}}{n^{1/2-2s}}\Big)
=
\omega \xi\sum_{i=1}^n X_i + o_P(1),
\end{align*}
and
\begin{equation*}
    \omega^2 \xi^2 \sum_{i=1}^n Y_i^2=n\omega^2 \xi^2 +o_P(1).
\end{equation*}

Then
\begin{align*}
    & \int_{A((\log n)^r)\cap \{\omega\geq n^{-s}\}} \Big[\prod_{i=1}^n \frac{p(X_i|\omega,\xi)}{p_0(X_i)}\Big]^t \pi(\omega,\xi)\, d\omega d\xi
    \\
    =&(1+o_P(1))\int_{A( (\log n)^r )\cap \{\omega\geq  n^{-s}\} } \exp\big\{
        t\omega \xi \sum_{i=1}^n X_i -\frac{1}{2} nt\omega^2 \xi^2
    \big\} \pi(\omega,\xi)\, d\omega d\xi.
\end{align*}
Note that on $A((\log n)^r)\cap \{\omega\geq n^{-s}\}$, $\pi_{\xi}(\xi)=(1+o(1))\pi_\xi(0)$. Then
\begin{align*}
    &\int_{A((\log n)^r)\cap \{\omega\geq n^{-s}\} } \exp\big\{
        t\omega \xi \sum_{i=1}^n X_i -\frac{1}{2} nt\omega^2 \xi^2
    \big\} \pi(\omega,\xi)\, d\omega d\xi
    \\
    =&(1+o_P(1))\pi_{\xi}(0)\int_{A((\log n)^r)\cap \{\omega\geq  n^{-s}\} } \exp\big\{
        t\omega \xi \sum_{i=1}^n X_i -\frac{1}{2} nt\omega^2 \xi^2
    \big\} \pi_{\omega}(\omega)\, d\omega d\xi
    \\
    =&(1+o_P(1))\pi_{\xi}(0)\int_{n^{-s}}^1 \pi_{\omega}(\omega)\, d\omega 
    \int_{-2\Phi^{-1}\big((\log n)^r/(2\omega \sqrt{n})+1/2\big)}^{2\Phi^{-1}\big((\log n)^r/(2\omega \sqrt{n})+1/2\big)} \exp\big\{
        t\omega \xi \sum_{i=1}^n X_i -\frac{1}{2} nt\omega^2 \xi^2
    \big\} d\xi.
\end{align*}
Note that
\begin{align*}
    &\int_{-2\Phi^{-1}\big((\log n)^r/(2\omega \sqrt{n})+1/2\big)}^{2\Phi^{-1}\big((\log n)^r/(2\omega \sqrt{n})+1/2\big)} \exp\big\{
        t\omega \xi \sum_{i=1}^n X_i -\frac{1}{2} nt\omega^2 \xi^2
\big\} d\xi
\\
=&
 \frac{1}{\omega} \sqrt{\frac{2\pi}{tn}}  \exp\Big\{\frac{t}{2n}(\sum_{i=1}^n X_i)^2\Big\}
\bigg[
    \Phi\bigg(2\sqrt{tn}\omega \Phi^{-1}\Big(\frac{(\log n)^r}{2\omega\sqrt{n}}+\frac 12\Big)-\sqrt{\frac t n} \sum_{i=1}^n X_i\bigg)
    \\
    &
    -
    \Phi\bigg(-2\sqrt{tn}\omega \Phi^{-1}\Big(\frac{(\log n)^r}{2\omega\sqrt{n}}+\frac 12\Big)-\sqrt{\frac t n} \sum_{i=1}^n X_i\bigg)
\bigg].
\end{align*}
Note that 
\begin{equation*}
    2\sqrt{tn}\omega \Phi^{-1}\Big(\frac{(\log n)^r}{2\omega\sqrt{n}}+\frac 12\Big)
    \geq 
    \sqrt{2\pi t} (\log n)^r.
\end{equation*}
It follows that
\begin{align*}
    &\int_{-2\Phi^{-1}\big((\log n)^r/(2\omega \sqrt{n})+1/2\big)}^{2\Phi^{-1}\big((\log n)^r/(2\omega \sqrt{n})+1/2\big)} \exp\big\{
        t\omega \xi \sum_{i=1}^n X_i -\frac{1}{2} nt\omega^2 \xi^2
\big\} d\xi
=
 \frac{1}{\omega}\sqrt{\frac{2\pi}{tn}}  \exp\Big\{\frac{t}{2n}(\sum_{i=1}^n X_i)^2\Big\}
(1+o_P(1)),
\end{align*}
where the $o_P(1)$ term is uniform for $\omega$.
Thus,
\begin{align*}
    &\int_{A((\log n)^r)\cap \{\omega\geq n^{-s}\} } \exp\big\{
        t\omega \xi \sum_{i=1}^n X_i -\frac{1}{2} nt\omega^2 \xi^2
    \big\} \pi(\omega,\xi)\, d\omega d\xi
    \\
%=&(1+o_P(1))\pi_{\xi}(0)\sqrt{\frac{2\pi}{tn}}\exp \Big\{ \frac{t}{2n}(\sum_{i=1}^n X_i)^2\Big\} 
%\int_{n^{-s}}^1 
%\frac{1}{\omega}
%\pi_{\omega}(\omega)\, d\omega
%\\
=&(1+o_P(1))\pi_{\xi}(0)\sqrt{\frac{2\pi}{tn}}\exp \Big\{ \frac{t}{2n}(\sum_{i=1}^n X_i)^2\Big\} 
\int_{0}^1 
\frac{1}{\omega}
\pi_{\omega}(\omega)\, d\omega.
\end{align*}


Now we consider $A((\log n)^r)\cap \{\omega\leq n^{-s}\} $. By Theorem 2 of \cite{LIU200461}, we have 
\begin{equation*}
    \sup_{\omega\in [0,1],t\in \mathbb{R}}
    \sum_{i=1}^n \big(\log p(X_i|\omega,\xi)-\log p_0(X_i)\big)
    =O_P(\log \log n).
\end{equation*}
Thus,
\begin{align*}
    & \int_{A( (\log n)^r )\cap \{\omega\leq n^{-s}\}} \Big[\prod_{i=1}^n \frac{p(X_i|\omega,\xi)}{p_0(X_i)}\Big]^t \pi(\omega,\xi)\, d\omega d\xi
    \\
    %=&\int_\Theta \exp\big\{
        %\omega \xi \sum_{i=1}^n Y_i -\frac{1}{2} \omega^2 n \big(\exp(\xi^2)-1\big)
    %\big\} \pi(\omega,\xi)\, d\omega d\xi\\
    =&\exp\big\{O_P(\log(\log n))\big\}
    \Pi\big(\omega(2\Phi(|\xi|/2)-1)\leq (\log n)^r n^{-1/2}, \omega\leq n^{-s}\big).
\end{align*}
We break the probability into two parts: 
\begin{align*}
    &\Pi\big(\omega(2\Phi(|\xi|/2)-1)\leq (\log n)^r n^{-1/2}, \omega\leq n^{-s}\big)
    \\
    \leq &
    \Pi\big(\omega(2\Phi(|\xi|/2)-1)\leq (\log n)^r n^{-1/2}, \omega\leq  2(\log n)^r n^{-1/2}\big)
    \\
    &
    +
    \Pi\big(\omega(2\Phi(|\xi|/2)-1)\leq (\log n)^r n^{-1/2},   2(\log n)^r n^{-1/2} \leq \omega\leq n^{-s}\big)
    .
\end{align*}
The first probability satisfies
\begin{align*}
    &\Pi\big(\omega(2\Phi(|\xi|/2)-1)\leq (\log n)^r n^{-1/2}, \omega\leq  2(\log n)^r n^{-1/2}\big)
    \\
    \leq &
    \Pi\big( \omega\leq  2(\log n)^r n^{-1/2}\big)
    \lesssim 
    \int_{0}^{2(\log n)^r n^{-1/2}} w^{\alpha_1-1}\, d\omega
    \lesssim \Big(\frac{(\log n)^r}{\sqrt{n}}\Big)^{\alpha_1}.
\end{align*}
Next we deal with the second probability.
On the event of the second probability, we have
$
    (2\Phi(|\xi|/2)-1)\leq \omega^{-1} (\log n)^r n^{-1/2}\leq 1/2
    $,
which implies the boundedness of $\xi$.
It follows that
$
|\xi|\leq C\omega^{-1} (\log n)^r n^{-1/2}
$
for some constant $C>0$ on this event.
Thus, on this event,
\begin{align*}
    &\Pi\big(\omega(2\Phi(|\xi|/2)-1)\leq (\log n)^r n^{-1/2},   2(\log n)^r n^{-1/2} \leq \omega\leq n^{-s}\big)
    \\
    \leq
    &\Pi\big(|\xi|\leq C \omega^{-1} (\log n)^r n^{-1/2},   2(\log n)^r n^{-1/2} \leq \omega\leq n^{-s}\big)
    \\
    \leq
    &\Pi\big(|\xi|\leq C \omega^{-1} (\log n)^r n^{-1/2},\omega\leq n^{-s}\big)
    \\
    \lesssim &
    \int_{0}^{n^{-s}} \omega^{\alpha_1-1}\, d\omega
    \int_{-C \omega^{-1} (\log n)^r n^{-1/2}}^{C \omega^{-1} (\log n)^r n^{-1/2}} \pi_{\xi}(\xi) \, d\xi.
\end{align*}
There exits $\epsilon>0$ and $M>0$ such that $\pi_{\xi}(\xi)\leq M$ for $\xi\in [-\epsilon,\epsilon]$. Then
\begin{align*}
    &\int_{0}^{n^{-s}} \omega^{\alpha_1-1}\, d\omega
    \int_{-C \omega^{-1} (\log n)^r n^{-1/2}}^{C \omega^{-1} (\log n)^r n^{-1/2}} \pi_{\xi}(\xi) \, d\xi
    \\
    \leq &
    \int_{0}^{{C(\log n)^r}/{(\epsilon \sqrt{n})}} \omega^{\alpha_1-1}\, d\omega
    +
    \int_{{C(\log n)^r}/{(\epsilon \sqrt{n})}}^{n^{-s}} 
    2MC \omega^{\alpha_1-2}(\log n)^r n^{-1/2}
    \, d\omega
    \\
    \lesssim &
     \Big(\frac{(\log n)^r}{\sqrt{n}}\Big)^{\alpha_1}
     +\frac{(\log n)^r}{\sqrt{n}}\Bigg(\Big(\frac{(\log n)^r}{\sqrt{n}}\Big)^{\alpha_1-1}\vee \Big(\frac{1}{n^{s}}\Big)^{\alpha_1-1}\Bigg)
     \\
    = &
    \Big(\frac{(\log n)^r}{\sqrt{n}}\Big)^{\alpha_1}\vee \frac{(\log n)^r}{n^{1/2+s(\alpha_1-1)}}.
\end{align*}
Thus,
\begin{align*}
    & \int_{A( (\log n)^r )\cap \{\omega\leq n^{-s}\}} \Big[\prod_{i=1}^n \frac{p(X_i|\omega,\xi)}{p_0(X_i)}\Big]^t \pi(\omega,\xi)\, d\omega d\xi
    \\
    %=&\int_\Theta \exp\big\{
        %\omega \xi \sum_{i=1}^n Y_i -\frac{1}{2} \omega^2 n \big(\exp(\xi^2)-1\big)
    %\big\} \pi(\omega,\xi)\, d\omega d\xi\\
    =&\exp\big\{O_P(\log(\log n))\big\}
    \Bigg(\Big(\frac{(\log n)^r}{\sqrt{n}}\Big)^{\alpha_1}\vee \frac{(\log n)^r}{n^{1/2+s(\alpha_1-1)}}\Bigg)=o_P(n^{-1/2}).
\end{align*}
Thus,
\begin{align*}
    & \int \Big[\prod_{i=1}^n \frac{p(X_i|\omega,\xi)}{p_0(X_i)}\Big]^t \pi(\omega,\xi)\, d\omega d\xi
    \\
    =& \bigg( \int_{A( (\log n)^r )^c\}}+\int_{A( (\log n)^r )\cap \{\omega< n^{-s}\}}+\int_{A( (\log n)^r )\cap \{\omega\geq n^{-s}\}}\bigg) \Big[\prod_{i=1}^n \frac{p(X_i|\omega,\xi)}{p_0(X_i)}\Big]^t \pi(\omega,\xi)\, d\omega d\xi
    \\
    %=&\int_\Theta \exp\big\{
        %\omega \xi \sum_{i=1}^n Y_i -\frac{1}{2} \omega^2 n \big(\exp(\xi^2)-1\big)
    %\big\} \pi(\omega,\xi)\, d\omega d\xi\\
    =&
(1+o_P(1))\pi_{\xi}(0)\sqrt{\frac{2\pi}{tn}}\exp \Big\{ \frac{t}{2n}(\sum_{i=1}^n X_i)^2\Big\} 
\int_{0}^1 
\frac{1}{\omega}
\pi_{\omega}(\omega)\, d\omega.
\end{align*}

Thus,
\begin{equation*}
    2\log \Lambda_{a,b}(\BX^{(n)})=-\log(1+a/b)+\frac{a}{n}(\sum_{i=1}^n X_i)^2+o_P(1).
\end{equation*}

%Now suppose
\end{proof}

\end{appendices}


\bibliographystyle{apa}
\bibliography{mybibfile}


\end{document}
